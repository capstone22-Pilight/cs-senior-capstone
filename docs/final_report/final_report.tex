\documentclass[10pt,draftclsnofoot,onecolumn]{IEEEtran}
\usepackage[margin=0.75in]{geometry}
\usepackage{listings}
\usepackage{underscore}
\usepackage[breaklinks]{hyperref}
\usepackage{url}
\usepackage{breakurl}
\usepackage{graphicx}
\usepackage{atbegshi}% http://ctan.org/pkg/atbegshi

\renewcommand{\refname}{References}
\lstset{ %
   language=Python,                % choose the language of the code
   basicstyle=\small,        % the size of the fonts that are used for the code
   keywordstyle=\color{blue},
   stringstyle=\color{red},
   commentstyle=\color{green},
   numbers=left,                   % where to put the line-numbers
   numberstyle=\footnotesize,      % the size of the fonts that are used for the line-numbers
   stepnumber=1,                   % the step between two line-numbers. If it is 1 e ach line will be numbered
   numbersep=5pt,                  % how far the line-numbers are from the code
   backgroundcolor=\color{white},  % choose the background color. You must add \usep ackage{color}
   showspaces=false,               % show spaces adding particular underscores
   showstringspaces=false,         % underline spaces within strings
   showtabs=false,                 % show tabs within strings adding particular unde rscores
   frame=single,           % adds a frame around the code
   tabsize=2,          % sets default tabsize to 2 spaces
   captionpos=b,           % sets the caption-position to bottom
   breaklines=true,        % sets automatic line breaking
   breakatwhitespace=false,    % sets if automatic breaks should only happen at whitespace
   escapeinside={\%*}{*)}          % if you want to add a comment within your codey
   }
\hypersetup{
    bookmarks=false, % show bookmarks bar?
    pdftitle={Spring Term Report Final Stage}, % title
    pdfauthor={Malcom Diller, Evan Steele, Sean Rettig}, % author
    pdfsubject={Raspberry Pi Outdoor Lighting}, % subject of the document
    pdfkeywords={TeX, LaTeX, graphics, images}, % list of keywords
    colorlinks=true, % false: boxed links; true: colored links
    linkcolor=blue, % color of internal links
    citecolor=black, % color of links to bibliography
    filecolor=black, % color of file links
    urlcolor=blue, % color of external links
  %linktoc=page % only page is linked
}
\title{Not exactly the Internet of Things for Outdoor Lighting\\ Spring Term Report\\ Final Stage}
\author{Oregon State University CS Senior Capstone Group 22\\Malcolm Diller, Sean Rettig, Evan Steele\\Client: Victor Hsu}

\begin{document}
\maketitle
\begin{abstract}
The goal of this project is to create a home lighting automation system from
open source software and inexpensive commodity hardware that allows lights to
be controlled automatically and remotely while also being easy to set up and
use.\\

The system consists of a wireless network of tiny ``client'' computers that
each control up to 4 sets of lights and are controlled by a central ``server''
computer, which automatically sends out commands to the clients when it's
time to turn on or off.  The central node runs a control program that can
be easily accessed via a touch screen, a web browser, or a mobile device, where
the user can locally or remotely control each light individually. The control
interface allows users to easily set ``rules'' for what their lights do and
when, depending on the time of day, the sun/moon position, and potentially even
triggers such as weather conditions or calendar dates.\\

Since September 2015, we have been developing the final deliverable through several 
iterations. The core system was completed for the Winter progress report in the beta
stage, in which all functionality was present. The Spring term was used to polish the 
end result and work on various bugs discovered through extensive testing. Our \href{https://github.com/rettigs/cs-senior-capstone}{Github repository} served as our issue tracker, and as the host for our codebase. 
The project was extensively tested and debugged before our final presentation.
\end{abstract}
\pagebreak
\tableofcontents
\newpage

\section{Status}
For the final release, all core components were completed and debugged,
leaving many quality-of-life changes for the Spring term. Our medium-priority issues
were resolved for the final
deliverable. Our full issue list can be 
found at \href{https://github.com/rettigs/cs-senior-capstone/issues}{Github}.

\begin{itemize}
    \item Allow option to simulate time running faster\\
	  - For testing purposes, we wanted to have a system where 
            we could pass a flag to the startup script and run an entire
            day in a usable time frame. 
    \item Better default names for lights\\ 
          - Early iterations had the light name start as "Light@<mac\_address>",
            which was confusing to read. Better names enhance the user experience.
    \item No login functionality\\
          - A login system was built into the application, however as development
            continued it was decided that it was unnecessary, given that we already
            authenticate to connect to the Pi as an access point.
    \item Jumbled display on mobile devices\\
          - Some of our web design choices were not performing well on mobile
            devices when the light names were long. More nesting in div elements
            fixed this problem.
    \item Server crash on invalid query\\
          - Early iterations of the query thread would cause an exception that would
            end the query runner too soon. Now we handle that exception and just skip 
            bad queries.
\end{itemize}

\section{Work To Do}

\section{Problems}

\section{Code}

\section{Concluding Analysis}

\newpage
\Urlmuskip=0mu plus 1mu\relax
\bibliography{design}{}
\bibliographystyle{IEEEtran}

\end{document}
