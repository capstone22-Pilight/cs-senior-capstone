\documentclass[10pt,draftclsnofoot,onecolumn]{IEEEtran}
\usepackage[margin=0.75in]{geometry}
\usepackage{listings}
\usepackage{underscore}
\usepackage[breaklinks]{hyperref}
\usepackage{url}
\usepackage{breakurl}
\usepackage{graphicx}
\usepackage{atbegshi}% http://ctan.org/pkg/atbegshi
\usepackage{color}
\usepackage{rotating}

\renewcommand{\refname}{References}
\definecolor{lightgray}{rgb}{.9,.9,.9}
\definecolor{darkgray}{rgb}{.4,.4,.4}
\definecolor{purple}{rgb}{0.65, 0.12, 0.82}

\lstset{
   language=Python,
   backgroundcolor=\color{lightgray},
   extendedchars=true,
   basicstyle=\footnotesize\ttfamily,
   showstringspaces=false,
   showspaces=false,
   numbers=left,
   numberstyle=\footnotesize,
   numbersep=9pt,
   tabsize=2,
   breaklines=true,
   showtabs=false,
   captionpos=b
}

\hypersetup{
    bookmarks=false, % show bookmarks bar?
    pdftitle={Final Report }, % title
    pdfauthor={Malcom Diller, Evan Steele, Sean Rettig}, % author
    pdfsubject={Raspberry Pi Outdoor Lighting}, % subject of the document
    pdfkeywords={TeX, LaTeX, graphics, images}, % list of keywords
    colorlinks=true, % false: boxed links; true: colored links
    linkcolor=blue, % color of internal links
    citecolor=black, % color of links to bibliography
    filecolor=black, % color of file links
    urlcolor=blue, % color of external links
  %linktoc=page % only page is linked
}
\title{Not Exactly the Internet of Things for Outdoor Lighting\\ Spring Term 2016 Report\\ Final Stage}
\author{Oregon State University CS Senior Capstone Group 22\\Malcolm Diller (dillerm@oregonstate.edu), Sean Rettig (rettigs@oregonstate.edu), Evan Steele (steelee@oregonstate.edu)\\Client: Victor Hsu (hsuv@onid.orst.edu)}

\begin{document}
\maketitle
\begin{abstract}

``Not Exactly the Internet of Things for Outdoor Lighting'' (or simply
``PiLight'') is a self-contained home lighting automation system designed to be
powerful and highly customizable, yet inexpensive.  Standard wallwart plug
timers are cheap, but their usefulness is limited.  High-tech
Internet-connected automation systems are feature-rich, but are often very
expensive, are not easily customizable, and can be rendered useless if the
external service is ever shut down.  PiLight solves this issue by not requiring
any external dependencies to function normally.  Furthermore, PiLight is built
from commodity hardware and open source software, so the system is not only
inexpensive, but also easy to extend and personalize, for the technically
inclined.  PiLight includes a sophisticated rule and scheduling system and has
the potential to integrate with devices other than lights, such as sprinkler
systems, garage door openers, and more.

\end{abstract}
\newpage
\tableofcontents
\newpage
\section{Introduction}
``Not Exactly the Internet of Things for Outdoor Lighting'' (or simply
``PiLight'') is a self-contained home lighting automation system designed to be
powerful and highly customizable, yet inexpensive.  Originally commissioned by
Victor Hsu at Oregon State University for simple outdoor lighting automation,
the project has expanded to include a sophisticated rule and scheduling system
and has the potential to integrate with devices other than lights, such as
sprinkler systems, garage door openers, and more.

While similar products do exist, PiLight attempts to fill a niche between
``overly simplistic'' and ``walled garden''.  Standard wallwart plug timers are
cheap, but their usefulness is limited; if you want your lights to turn on at
sunset, for example, you must manually adjust the timer throughout the year.
If you only want your lights on for the weekends, you're out of luck.
High-tech Internet-connected automation systems are feature-rich, but are often
very expensive and not easily customizable.  Even worse, should the external
Internet services running the show break or shut down, the entire system can be
rendered useless.  PiLight solves these issues by being self-contained; no
external dependencies are required for the system to function normally.
Furthermore, PiLight is built from commodity hardware and open source software,
so the system is not only inexpensive, but also easy to extend and personalize,
for the technically inclined.

PiLight consists of a wireless network of tiny ``client'' computers that each
control up to 4 sets of lights and are controlled by a central ``server''
computer, which automatically sends out commands to the clients when it's time
to turn on or off.  The central node runs a control program that can be easily
accessed via a touch screen, a web browser, or a mobile device, where the user
can locally or remotely control each light individually. The control interface
allows users to easily set ``rules'' for what their lights do and when,
depending on the time of day, day of the week, sun position, and potentially
even triggers such as weather conditions or calendar dates.

A Raspberry Pi with a small touchscreen and Wi-Fi card comprises the central
control unit; it runs its own web server to allow control of lights, as well as
a wireless network that allows both users and client nodes to connect.  The
client nodes are each composed of an ESP8266 Wi-Fi module and a relay, which
automatically connect to the Pi's wireless network when the ``Add Devices''
button is pressed on the web UI.  In a commercial implementation of PiLight,
the client nodes could be integrated into a standard power strip/surge
protector, so users can simply plug their devices in and start using the system
immediately without having to order and wire up the components individually.

The PiLight team is composed of Malcolm Diller, Sean Rettig, and Evan Steele,
who are computer science students at Oregon State University.  Evan Steele's
primary role was creating the Yocto Linux builds for the Raspberry Pi and
writing the ESP8266 firmware, while Malcolm and Sean focused primarily on the
web application, a Flask app with a Bootstrap frontend.  In particular, Sean
developed most of the rule system and advanced settings for lights, while
Malcolm and Evan developed the front page UI, using Javascript to dynamically
update the statuses and names of lights.  Throughout development, we have met
with our client, Victor, to help clarify project requirements, drive
hardware/software design decisions, and provide general guidance to ensure the
project's success.

\section{Original Requirements Document}
\input{original-requirements}
\section{Updated Requirements}
Since the original requirements document, only three of our requirements have
been updated, and all for the same reason; they were made into stretch goals
due to hardware limitations.

\begin{tabular}{ l p{7cm} p{2cm} p{6cm} }
    & \textbf{Requirement} & \textbf{Result} & \textbf{Comments} \\
    1f & Lights can be toggled over a user-specified period of time, e.g. a light can gradually turn on and grow brighter over the course of 30 seconds. & Changed to stretch goal. & We decided that this functionality was not possible with the provided relays, which can only toggle lights on or off, not vary their intensity.\\
    2i & The web interface contains sliders for each light that can be used to change the intensity of lights individually. & Changed to stretch goal. & We decided that this functionality was not possible with the provided relays, which can only toggle lights on or off, not vary their intensity.\\
    3m & Lights/groups can be set to gradually dim/brighten over a set period of time.  For example, lights can turn on and gradually brighten from sunset to 30 minutes after sunset, after which they are at full brightness. & Changed to stretch goal. & We decided that this functionality was not possible with the provided relays, which can only toggle lights on or off, not vary their intensity.\\
\end{tabular}

\subsection{Updated Gantt Chart}

Gantt chart available at our project SharePoint website:
%\url{https://oregonstateuniversity-my.sharepoint.com/personal/rettigs_oregonstate_edu/capstone22_project/_layouts/15/start.aspx#/Lists/Tasks/gantt.aspx}
%\begin{sidewaysfigure}
%\includegraphics[width=1.0\textwidth]{ganttchart-original.png}
%\end{sidewaysfigure}

\section{Design Document}
\subsection{Introduction}

In the twenty-first century, your home should be smart and responsive. An
automated home should be able to control appliances such as lights through
single-board computer units that have Internet connectivity. Many commercial
options exist, such as those from Nest Labs, but such devices are prohibitively
expensive for most consumers, especially for simple tasks such as lighting
automation. This project aims to provide an low-cost alternative for simple
home automation, for devices like lights and garage door openers. Using a
low-cost embedded computer such as the Raspberry Pi and cheap wireless
communication modules such as the ESP8266, this project will provide a cheap,
easy-to-install home automation system.

The project is called "Not Exactly the Internet-of-Things" because it's not
designed as a typical IoT product. The network is entirely local, not worrying
about an external connection to the Wide Area Network. Some additional
components such as consulting an external API could be considered, but that's
an addon to the finished product. As it is made as a home application, it is
readily accessible through smartphones and web interfaces. With a clean user
interface that enables both manual control and automation profiles, the
application will require a minimal learning curve and be simple to set up.  

\subsection{Design View}

The aim of this project is to create a home lighting automation system that is
inexpensive, wireless, and easy to use. The user of this product should the
following features:

\begin{enumerate}
    \item Lights and Groups
    \begin{enumerate}
        \item The lights can be controlled from the built-in touchscreen on the
            control unit, or a mobile device such as a phone, tablet, or laptop
        \item Lights can be put into "groups" so that they can be controlled
            all at once
        \item Groups can also be nested into other groups to create hierarchies
            of lights, in order to control many lights at once
        \item The user will be able to nickname lights and light groups
        \item The intensity of each light can be adjusted using a slider
    \end{enumerate}
    \item Rules / Types of Control
    \begin{enumerate}
        \item Lights can be toggled on and off manually, but they can also be
            set to be activated on specific schedules using "rules"
        \item Rules can be set for lights or groups of lights
        \item There are 6 main types of schedules that rules can trigger on
        \begin{enumerate}
            \item Time of day (e.g. “turn on at 8pm and turn off at 6am”) 
            \item Day of week (e.g. “turn on during Wednesdays”)
            \item Day of month (e.g. “turn on every 1st of the month”)
            \item Day of year (e.g. “turn on every January 1st”)
            \item Specific dates (e.g. “turn on from Dec. 24th at 2am to Dec
                27th at 8pm”)
            \item Sunrise / sunset (e.g "turn on when the sun sets, turn off
                when the sun rises")
        \end{enumerate}
        \item Multiple rules can be applied to each light or group of lights by
            using AND or OR logic
        \item Lights can be toggled based on the toggle state of its parent
            group
        \item Rules can be enabled or disabled temporarily
        \item Lights or groups can be set to gradually dim/brighten over a set
            period of time instead of toggling
    \end{enumerate}
    \item Stretch Goals
    \begin{enumerate}
        \item The system can control devices other than just lights:
        \begin{enumerate}
            \item Garage door openers
            \item Music players
            \item Holiday decorations
            \item Sprinkler systems
        \end{enumerate}
        \item The web interface is accessible over the Internet (Instead of
            only on the local network)
        \begin{enumerate}
            \item The interface is accessible for only the homeowner or
                authorized individuals
            \item The interface hosted on a remote server if the user does not
                want to deal with port forwarding their home network.
        \end{enumerate}
        \item Additional types of triggers for rules
        \begin{enumerate}
            \item Weather conditions (e.g. lights can turn on automatically if
                it starts raining)
            \item Schedule in a Google calendar
            \item Moon position or other celestial data (e.g. lights could turn
                off during a full moon or solar eclipse for better viewing)
            \item Motion sensors (e.g. lights turn on when someone walks up to
                the front door)
            \item Light sensors (e.g. lights turn off at a certain brightness
                level)
            \item Moisture sensors (e.g. lights turn on when it's wet outside)
            \item Sound sensors (e.g. lights turn on after a loud noise, or
                strobe to the beat of music that is playing)
        \end{enumerate}
    \end{enumerate}
\end{enumerate}

\subsection{Design Viewpoints}

We will be using a wide variety of open source technologies and inexpensive
hardware to provide a low-cost home automation system. We will be using a
varied set of tools to accomplish our goal, including:

\begin{enumerate}
    \item The Yocto Project 
    \begin{enumerate}
        \item The Yocto Project is a kernel building system designed for
            embedded systems. It easily incorporates additional features and
            custom packages directly from Git repositories. It makes it easy to
            build a customized image with only exactly what we want and no
            unnecessary packages \cite{yocto}. It also allows us to customize elements of
            the kernel such as startup scripts in init.d, kernel modules, and
            network scripts \cite{flask}. The kernel will be compiled for the Raspberry Pi
            and copied to an SD card using Yocto's filesystem script.
    \end{enumerate} 
        \item Raspberry Pi 2
    \begin{enumerate}
        \item The Raspberry Pi 2 is an ARMv7-processor-based microcomputer
            designed for educational and embedded projects \cite{raspi}. We will use it as a
            low-cost server solution for the lighting automation system. It can
            serve as the web server running NGINX and transmit TCP commands to
            the ESP8266 wifi modules \cite{nginx}. The Pi will be outfitted with a 320x280
            TFTLCD display \cite{tftlcd} so that the interface for control can be
            accessed directly.  The Pi will also act as a wireless router for
            the plug units to connect to, and run a DHCP server to distribute
            IP addresses to the plug units.
    \end{enumerate} 
        \item ESP8266
    \begin{enumerate}
        \item The ESP8266 is a self-contained SoC (System on a Chip) that has a
            built-in Wi-Fi radio \cite{esp8266}. Our module is outfitted with custom firmware,
            written in Lua and flashed to the device using the Arduino IDE,
            that allows the module to automatically connect to the Raspberry Pi
            (control unit) that acts as a WAP (Wireless Access Point) \cite{lua}. The
            board we have is designed to connect to an electronic relay using
            jumper cables. Using simple TCP commands, the board will flip the
            relay pins on and off \cite{relay}.
    \end{enumerate}
\end{enumerate}

\subsection{User Experience}

The final deliverable of a commercial version of this product would likely
include the following:

\begin{itemize}
    \item A central control unit consisting of the Raspberry Pi, its power
        cable, the touchscreen module, a case, and an SD card with the server
        software preloaded onto it.
    \item One or more plug units which would likely resemble a power strip,
        with each plug being individually controllable by the system.
\end{itemize}

To set up the system, the user simply plugs both the control unit and the plug
unit(s) into wall power.  The plug units will preconfigured to connect to the
control unit.  The user can then plug devices into the plug units, such as
lamps, and the display on the control unit will update itself automatically
with a list of lights.  The lights can have either numbered (e.g. "Zone 1,
Light 1") or randomized but memorable names by default (e.g. "Blue Koala")
which can be later edited through the web interface.  To access the web
interface, the user connects to the wifi network provided by the control unit
using the credentials included with the manual, or perhaps printed on the
control unit itself.  They then open a web browser and connect to a specific
URL, provided to the user with the wifi password.  From the web interface, all
features are accessble, such as renaming lights, creating light groups, and
applying rules for when to toggle lights.

\subsection{Testing}

To test the functionality and correctness of our system, we will test the
various components both individually and when integrated with other components:

\begin{itemize}
    \item We plan to test the wireless communication capabilities of the system
        by attempting to send packets between the control unit and plug units
        at varying distances and in varying environments, including open and
        walled areas.  Ideally, the range should be great enough to fully cover
        the average American house/apartment.  Once this is complete, it will
        likely not need to be changed or extended much, reducing the need for
        regression testing.
    \item We plan to test the actual light toggling (as performed by the plug
        units) manually.  Once this is complete, it will likely not need to be
        changed or extended much, reducing the need for regression testing.
    \item We plan to test the functionality of the touchscreen interface
        through manual human interaction testing.  Once this is complete, it
        will likely not need to be changed or extended much, reducing the need
        for regression testing.
    \item We plan to test the functionality of the web interface through a
        combination of manual human interaction testing, automated API-driven
        testing, and automated GUI testing (using a platform such as Selenium).
        The plug units can send back acknowledgement packets to confirm what
        actions they took in response to the tests, allowing nearly complete
        integration tests to be performed automatically when changes are made
        to the control program and web interface.  These automated tests will
        help prevent regressions as new features are developed and save time in
        ensuring the correctness of the system.
\end{itemize}

\subsection{Timeline}
{\renewcommand{\arraystretch}{0.8}
\begin{tabular}{ | l | l | }
   \hline
   \textbf{Device control} & \textbf{Due Date} \\ \hline

   Server can boot with GUI and touchscreen support & Thu 11/19/15 \\ \hline
   Server starts control program on power on & Mon 11/23/15 \\ \hline
   Clients are loaded with client control program & Thu 11/12/15 \\ \hline
   Clients discover connected relays/lights & Thu 11/19/15 \\ \hline
   Clients can toggle lights individually & Tue 11/24/15 \\ \hline
   Lights can be toggled over time & Fri 11/27/15 \\ \hline
   Server acts as wireless access point for clients & Mon 11/30/15 \\ \hline
   Clients can be paired with servers & Fri 12/4/15 \\ \hline
   Clients tell server which lights are available & Tue 12/8/15 \\ \hline
   Server can send light instructions to clients & Thu 12/10/15 \\ \hline
   Clients can receive light instructions from server & Mon 12/14/15 \\ \hline
   Clients can perform light instructions from server & Wed 12/16/15 \\ \hline
   
   \textbf{User Interface} & \\ \hline
   
   Server serves web site with control interface for user & Mon 11/30/15 \\ \hline
   Accessible via the system’s built-in touchscreen & Mon 12/7/15 \\ \hline
   Accessible to other devices like phones & Thu 12/3/15 \\ \hline
   Displays list of connected clients & Mon 12/7/15 \\ \hline
   Can nickname clients & Tue 12/8/15 \\ \hline
   Displays list of connected lights & Wed 12/9/15 \\ \hline
   Can nickname lights & Thu 12/10/15 \\ \hline
   Buttons to toggle lights individually & Thu 12/10/15 \\ \hline
   Sliders to change light intensity & Fri 12/11/15 \\ \hline
   Can put lights into groups & Mon 12/14/15 \\ \hline
   Displays list of groups & Tue 12/15/15 \\ \hline
   Can nickname groups & Wed 12/16/15 \\ \hline
   Groups can be nested & Wed 12/23/15 \\ \hline
   Can toggle lights/groups based on rules & Fri 1/1/16 \\ \hline
   
   \textbf{Rules} & \\ \hline
   
   Lights can be toggled manually, overriding any rules & Tue 12/15/15 \\ \hline
   Rules can be temporarily disabled & Wed 1/6/16 \\ \hline
   Lights can be toggled based on time of day & Tue 1/12/16 \\ \hline
   Lights can be toggled based on day of week & Thu 1/14/16 \\ \hline
   Lights can be toggled based on day of month & Thu 1/14/16 \\ \hline
   Lights can be toggled based on day of year & Thu 1/14/16 \\ \hline
   Lights can be toggled based on specific dates & Thu 1/14/16 \\ \hline
   Lights can be toggled based on sunrise/sunset times & Fri 1/22/16 \\ \hline
   Multiple rules per light/group & Fri 1/15/16 \\ \hline
   Lights can be toggled based on parent group & Tue 1/5/16 \\ \hline
   Rules can toggle early/late by random time & Tue 1/5/16 \\ \hline
   Lights can toggle early/late randomly within groups & Wed 1/6/16 \\ \hline
   Lights can gradually toggle over time & Thu 1/7/16 \\ \hline
\end{tabular}
}
\subsection{Changes to the Design Document}

Over the course of the year, there were a few changes to the project that had to be made to allow for new problems that came up when implementing some of the features specified in the document. The only change that we have made to the design document since it was created was that we had to remove the lines that stated "The intensity of each light can be adjusted using a slider," and "Lights or groups can be set to gradually dim/brighten over a set period of time instead of toggling." This had to be done because relays only allow us to have two states, and unless we used a different implementation and way to connect to the lights, there is physically no way for us to do dimming or brightening.

\section{Technology Review}
\subsection{Client}

Victor Hsu\\
Oregon State University\\
Phone: 541-737-4398\\
Email: hsuv@onid.orst.edu

\subsection{Problem Statement}

Outdoor lighting seems like a simple problem to solve, but the solutions on the
market today are less than ideal.  The standard transformer/timer combos
available in the big box stores are rudimentary and clunky at best (constantly
needing to be adjusted for the changing sunset and sunrise times), but are
reasonably priced. The new-generation smart apps for home automation are
flexible and fancier, but are quite spendy and lock you into a specific
protocol.  So why not use an open platform running on commodity hardware?  Easy
to use, reasonably priced, and highly customizable--that is our goal. \\

\subsection{Project Description}

Our system will consist of a wireless network of tiny “client” computers that
each control up to 4 sets of lights and are controlled by a central "server"
computer, which will automatically send out commands to the clients when it's
time to turn on or off.  The central node will run a control program that can
be easily accessed via a touch screen, a web browser, or a mobile device, where
the user can locally or remotely control each light individually.  Want your
lights to turn on at sunset and then dim gradually as the sun rises?  Simple.
Want your lights to flash when you're throwing a party?  Just press a button.
The control interface will allow users to easily set "rules" for what their
lights do and when, depending on the time of day, the sun/moon position, and
potentially even triggers such as weather conditions or calendar dates.  This
system will be easily extensible to potentially control other devices as well,
such as garage doors, sound systems, and more. \\
\hypersetup{linkcolor=blue}
\subsection{Pieces}

\begin{enumerate}
    \item OS for the PI
        \begin{enumerate}
            \item Raspbian \\
                Custom built for the PI, this operating system would provide many important features, and be a solid foundation to run on. Installation is very straightforward, and it has support all across the Internet. This means that the Pi would have to run the entire Raspbian operating system, so there will be many utilities we don't need that will be included anyway. The image is easily acquired and it can be burned to a SD card and booted straight away. It will likely require heavy configuration to get all the necessary services running and the extra ones to never run.
            \item Yocto meta-recipe \\
                We will not need all of the fancy features that Raspbian offers, and one way we can strip our OS down to the features we need is by building a Linux system using Yocto. The Yocto build system is complicated, but our group has experience building layers for various images. It will allow us to build a custom distribution with every package we need, init.d scripts, and service scripts for only the services we need. It will require build time, and time to burn to installation media. Kevin noted that we could get a build server for the project by the end of the term, which would give us good hardware for Yocto builds. 
            \item OpenWRT \\
                An open source router firmware package that allows embedded devices to perform complicated network tasks. Only recently was it recompiled to work with the Raspberry Pi. Our collective experience with the firmware is limited, so it would have an exceptional learning curve. It looks like it would be able to handle all the services we would need, but there isn't a whole lot of flexibility for other additions. It's also very difficult to debug or run tests with. 
        \end{enumerate}
        We have decided to use a Yocto build of Linux for the final build. This decision was made since the Yocto build will only contain exactly what we need, will allow us to easily incorporate new packages with the recipe system, and can be easily shared on our Github repository as a meta-layer for others to build. Proof of concept builds will just use Raspbian with the packages installed.
    \item Server Program Implementation
        \begin{enumerate}
            \item Python / MicroPython \\
            The ESP8266 can run MicroPython for GPIO access, making Python the obvious choice for the server program as well. Using Python's socket API, we can easily connect the devices and execute code to work with the GPIOs. The overhead of Python is minimal when considering the ease of use, and MicroPython was practically designed for the ESP8266, so there's lots of support available. Installing it is as simple as flashing a binary to the ESP8266, and the Python shell is supported. However, we would likely load in a Python script to handle incoming commands from the server. The ESP8266 has sufficient storage for one large Python script.
            \item C / libmraa \\
            Running the system through C is possible, thanks to \href{https://github.com/intel-iot-devkit/mraa}{libmraa} which provides useful abstractions from the /sys/class/gpio and puts it into easy-to-use C libraries. While we could use these libraries, the Python variant provides an easier interface with only slightly more overhead. Additionally, the library contains SWIG-generated wrappers for a variety of languages, including Python, C++, and Perl. The repository is maintained by Intel's IoT Development team, and our team has contribution experience with libmraa. 
            \item sysfs \\
            The most direct, but perhaps least reliable method is to just use the sysfs interface. This involves doing things like echoing values into files and reading the raw files for values. For instance, accessing the SPI interface for a device registered with major number 0 would look like: \verb|cat /sys/class/spi_master/spi0/spi0.0/iio:device0| which is not ideal or clean. Additionally, it could break if we try to use it on systems with different sysfs layouts. It would be useful to use sysfs for testing, but we should not use bash scripts like this for our final product. 
        \end{enumerate}
        We have decided to use Python and MicroPython because of compatibility issues and ease-of-use for the devices. Python will run on both the ESP8266 and the Raspberry Pi with all the libraries we need to run the software. Our concerns about overhead are minuscule since the Python will be executing very simple code; it is just interacting with a relay switch. It will mesh well with our web interface, so we'll be using it as a CGI script in the web server.
    \item Web Site Implementation
        \begin{enumerate}
            \item JavaScript \\
            An easy language to work with, Javascript will allow us to use many different, responsive templates such as react.js and node.js for control. Node even has wrappers for sysfs functionality which may even allow us to deploy it on the ESP8266 devices for remote control. Using JS for the web interface would offload the majority of work from the Pi to the web browser, which could be useful depending on the complexity of our interface. If we end up using a GPIO abstraction library such as libmraa, it should be noted that most of them have wrappers to Javascript code.
            \item Django \\
            A web framework known best for rapid deployment, we could examine this technology as a way to power our entire web interface and backend system. It would have a longer ramp-up time since none of our group members are fully comfortable with Django. However, if we decide that the ramp-up time is worth it, we could see this platform as an all encompassing management tool. Django is well-supported, but does have large learning curve and would require exhaustive group research to understand the entire system. 
            \item PHP \\
            The hypertext preprocessor is, at first glance, perhaps not the best system to run our backend off of, but we're considering it for a few reasons. It can execute our CGI scripts with the mod\_php module for the web server, our group has significant working experience with the language, and it just works out of the box. It may require additional configuration, but from a 10,000 foot view it accomplishes all the tasks we need. It can mesh well with a wide variety of other technologies we may end up using, and is highly modular.
        \end{enumerate}
        We have decided to use the HTML/JS/PHP stack for our project, which means we'll be building a system from the ground up. We can point the Javascript to the CGI files to execute the Python that will change the relays at the remote end, use PHP for the web backend and database access, and use HTML for the pages themselves (although this one will likely just be generated through the PHP code too) to create our full web server stack. This does not rule out us using other technologies such as node or Django as a subsystem.
    \item Server/Client Communication
        \begin{enumerate}
            \item AD-HOC \\
            In this communication method, the devices are configured for AD-HOC mode, or a packet-radio system. The wireless adapters need to be able to support AD-HOC mode and be set to a specific channel and IP range. Given the difficulty we've had with AD-HOC networks in the past, we'll maybe try to avoid this mode, but it would be useful as a proof-of-concept for socket communication. However, this is likely not a long-term solution.
            \item Central WAP \\
            We would configure one Raspberry Pi as a central Wireless Access Point and have the ESP8266 units connect to it as wireless clients. The advantage to this mode is that we could use the Raspberry Pi to connect to the Internet to perform tasks such as talking to the Wunderground API, which is not possible in AD-HOC mode. This would require that the Pi run a DHCP server, along with other routing services.
            \item CAS \\
            A Central Authority Service model leans more toward the well-defined Internet of Things model. All of our devices rely on an external server for almost all their actions. It would coordinate between the devices, actually store all the data and perform all the heavy lifting, relying on the Pi and ESP8266 devices only for GPIO access. A drawback to this model is the additional server we'd need to configure and rely on, especially since the system would stop working if the server ever becomes inaccessible.
        \end{enumerate}
        We have decided to use the WAP model because it provides the benefits of the AD-HOC and CAS models without all the additional configuration burdens. A Yocto meta-recipe for the Pi's WAP functionality would be simple, and connections between the ESP8266 devices would become a trivial task. This does mean the Pi has to run extra services (like DHCP), but it should be more than capable of running the limited services we will require of it.
\end{enumerate}

\subsection{Changes to the Technology Review}

The only large change that we made in regard to the decisions in our technology review was that for the website, we used no PHP to modify the database, and instead used a python Object Relational Mapper (ORM) called SQLAlchemy. Additionally, we used a python web framework for our website called Flask. Combined, these allowed for much cleaner code and a much easer to manage database access.

\section{Blog}
\subsection{Fall Update 1}
Our first week of the capstone project "Not exactly the Internet of Things for Outdoor Lighting" was relatively uneventful, as we spent most of the time starting to meet with the client and coordinating with the group for meeting times.\\
Plans for the coming week:
\begin{itemize}
\item Start to work with the hardware as soon as we get it. This will probably consist of booting the Raspberry Pi with an operating system and compiling MicroPython on the Wifi Modules
\item if we get far enough, work on communication between the Pi and the controllers
\end{itemize}
Progress since last week:
\begin{itemize}
\item We got our project and met with the client
\item We finished our proposal document
\end{itemize}
No major issues this week\\
Blog Date: 10-15-2015

\section{Poster}
\newpage
This page to be replaced with a full-page color printout of our poster.
\newpage
\section{The Project}
\subsection{Design}
\subsection{System Overview}
This project is divided into several sections that can be easily described through a graphic. Below is a block diagram that lays out the basic structure of the system.\\
\includegraphics[width=1.0\textwidth]{block-diagram.png}\\
As the diagram shows, the lights are connected to the back end of the server through the esp devices. These devices are configured to wirelessly connect to the server and recieve commands to turn the lights on or off. Each ESP device connects to a relay which can serve up to four different lights. The server then hosts a website that is used to control the lights and the schedules. This website can be viewed by any device that is able to connect to a wifi network, such as a laptop, smartphone, or tablet. It is also viewable from the touchscreen on the Pi itself.
\subsection{Theory of Operation}
Pilight should provide a way to more easily manage the lights in your home and provide an easy way to schedule times for them to turn on and off. 
\subsection{Installation}
As part of our installation, we have two modes of putting the image on your Raspberry Pi. These are tailored to users who just want an express installation with no difficulties, and for users who wish to build their own Pilight image from scratch using the Yocto Builder.
\subsection{Express install}
The wiki on Github has a link to a prebuilt image of the Pilight system we had in place at expo. This image is updated using Yocto's autobuilder program Toaster. Once downloaded, it can be installed like any other image for a Raspberry Pi.
\subsubsection{Windows}
Windows will use the graphical utility Win32DiskImager to install our image to the SD card.
\begin{itemize}
   \item Download the image from the Pilight Wiki at \url{https://github.com/rettigs/cs-senior-capstone/wiki}
   \item Insert the SD card into your SD card reader and check which drive letter was assigned. You can easily see the drive letter, such as G:, by looking in the left column of Windows Explorer. You can use the SD card slot if you have one, or a cheap SD adapter in a USB port.
   \item Download the Win32DiskImager program from \url{http://sourceforge.net/projects/win32diskimager/}, which will provide a way to put the image on the SD card
   \item Extract the executable from the zip file and run the Win32DiskImager utility; you may need to run this as administrator. Right-click on the file, and select Run as administrator.
   \item Select the image file you extracted earlier.
   \item Select the drive letter of the SD card in the device box. Be careful to select the correct drive; if you get the wrong one you can destroy the data on your computer's hard disk! If you are using an SD card slot in your computer and can't see the drive in the Win32DiskImager window, try using an external SD adapter.
   \item Click Write and wait for the write to complete.
   \item Exit the imager and eject the SD card.
\end{itemize}
\subsubsection{OSX}
On Macs we can use the command-line utilities for creating our image.
\begin{itemize}
   \item Insert your SD card to an SD card reader
   \item Use the disk reader program to identify your SD card:
      \begin{lstlisting}
      diskutil list
      \end{lstlisting}
   \item Identify the disk (not partition) of your SD card e.g \textit{disk4}, not \textit{disk4s1}
   \item Unmount the SD card
      \begin{lstlisting}
      diskutil unmountDisk /dev/disk<disk# from diskutil>
      \end{lstlisting}
      where \textit{disk} is your BSD name e.g. \textit{diskutil unmountDisk /dev/disk4}
   \item Copy the data to the SD card:
      \begin{lstlisting}
      sudo dd bs=1m if=pilight.img of=/dev/rdisk<disk# from diskutil>
      \end{lstlisting}
      where \textit{disk} is your BSD name e.g. \textit{/dev/disk4}
      \begin{itemize}
         \item This may result in a \textit{dd: invalid number '1m'} error if you have GNU coreutils installed. In that case, just change the \textbf{bs=1m} to \textbf{bs=1M}
      \end{itemize}
\end{itemize}
\subsubsection{Linux}
Our Linux install is similar to OSX, but using Linux tools instead
\begin{itemize}
   \item Run \textit{df} to find where your SD card is mounted:
      \begin{lstlisting}
      df -h
      \end{lstlisting}
      It will be listed as something like \textit{/dev/mmcblk0p1} or \textit{/dev/sdd1}. For later steps, you'll need the name without the partition number, which would change the values to \textit{/dev/mmcblk0} and \textit{dev/sdd}
   \item Unmount the drive. For \textit{/dev/sdd} it would be:
      \begin{lstlisting}
      umount /dev/sdd1
      \end{lstlisting}
   \item Write the data to the SD card. Again for \textit{/dev/sdd} it would be:
      \begin{lstlisting}
      dd bs=4M if=pilight.img of=/dev/sdd
      \end{lstlisting}
   \item Run sync to flush the write cache
      \begin{lstlisting}
      \end{lstlisting}
\end{itemize}
\subsection{Yocto Build}
Using the Yocto autobuilder allows you to build the image from scratch, however it does require some advanced knowledge of the Linux command line. You'll also need about 30 GB of drive space for the build.
\begin{itemize}
\item Start by fetching the builder and all the source repositories we'll need
\begin{lstlisting}
mkdir source
cd source
git clone -b dizzy git://git.yoctoproject.org/poky
cd poky
git clone -b dizzy git://git.openembedded.org/meta-openembedded
git clone -b dizzy git://git.yoctoproject.org/meta-intel-iot-middleware
git clone -b dizzy https://github.com/Pilight/meta-pilight.git
\end{lstlisting}
\item Initialize the build environment
\begin{lstlisting}
source oe-init-build-env
\end{lstlisting}
\item Now add the layers to the bblayers file and the configuration options to the local.conf file:
\begin{lstlisting}
echo "BBLAYERS += \"$HOME/source/poky/meta-pilight\"" >> conf/bblayers.conf
echo "BBLAYERS += \"$HOME/source/poky/meta-openembedded/meta-python\"" >> conf/bblayers.conf
echo "BBLAYERS += \"$HOME/source/poky/meta-openembedded/meta-oe\"" >> conf/bblayers.conf
echo "BBLAYERS += \"$HOME/source/poky/meta-openembedded/meta-networking\"" >> conf/bblayers.conf
echo "MACHINE = \"raspberrypi2\"" >> conf/local.conf
echo "CORE_IMAGE_EXTRA_INSTALL += \"kernel-dev\"" >> conf/local.conf
echo "CORE_IMAGE_EXTRA_INSTALL += \"pilight\"" >> conf/local.conf
echo "MACHINE_FEATURES_BACKFILL_CONSIDERED += \"rtc"\" >> conf/local.conf
echo "EXTRA_IMAGE_FEATURES = \"debug-tweaks package-management dev-pkgs tools-sdk dev-pkgs\"" >> conf/local.conf
\end{lstlisting}
\item Now build your image. This could take several hours depending on the system:
\begin{lstlisting}
bitbake maker-image
\end{lstlisting}
\item Using Yocto's builtin installer script, we can write the image to the SD card. Assuming it's at \textit{/dev/sdd}:
\begin{lstlisting}
$ sudo poky/scripts/contrib/mkefidisk.sh \
/dev/sdd \
poky/build/tmp/deploy/images/raspberrypi2/pilight-image-raspberrypi2.hddimg \
/dev/sda
\end{lstlisting}
We now have a bootable SD card
\end{itemize}
\section{New Technology}
There were several technologies we used that many people in our group were not familiar with, but learning them was not too great of a challenge.

\begin{itemize}
	\item Python \\
	While we had all had experience with python before starting on this project, for most of us this was our first large scale project that was almost entirely Python-based. For most questions we had about syntax particulars, we used the standard \href{www.docs.python.org}{Python Documentation}.
	\item SqlAlchemy \\
	For database managment, we used a Object Relational Mapping (ORM) python library called SqlAlchemy. For information on how to implement it and use it accordingly, we used the \href{www.docs.sqlalchemy.org}{SqlAlchemy Docs} on the SqlAlchemy Website.
	\item Flask \\
	We decided to use Flask as a framework for our website, which ended up being extremely useful, as it allows for a very seamless integration between the front and back ends of our server. As none of us had used Flask before, the \href{www.flask.pocoo.org\/docs}{Flask Documentation} was extremely helpful. Another resource that was useful for learning Flask in a more in-depth sense was \href{blog.miguelgrinberg.com\/post\/the-flask-mega-tutorial-part-i-hello-world}{The Flask Mega-Tutorial}.
	\item Bootstrap \\
	While some people in the group had used bootstrap before, others were new, and even those who were already familiar with it sometimes needed to look up specifics about how to use it. A resource we found useful was the main \href{www.getbootstrap.com}{Bootstrap Documentation}, as well as the \href{www.bootstrap-switch.org}{Bootstrap Switch} page for information on how to use the Switches plugin.
	\item Jquery \\
	The Jquery API proved very useful when writing the javascript for our website. As some of us were new to it and others had used it a decent amount, the \href{www.api.jquery.com}{Jquery Documentation} proved quite useful, especially when using the query functionality.
	\item Yocto \\
	Only one of our group members had used the Yocto build environment before, but he was able to provide the advice that we needed to make our project work.
	\item Firmware \\
	We needed to write firmware to make the ESP 8266 module do what we want. This would have been very obscure and hard to do without a reference of some sort, and the one we found useful was a \href{www.github.com\/nodemcu\/nodemcu-firmware}{github repository} that was fairly well organized. We based our firmware off of this code with some slight modifications.
	\item Google \\
	As always when learning new technology, google proved to be an invaluable resource, as the ability to search the web was very useful indeed.
\end{itemize}
\section{Team Discussion}
\subsection{Malcolm Diller}
\subsection{Sean Rettig}
\subsection{Evan Steele}
My experience with the Pilight project gave me the opportunity to merge my existing technical knowledge with an application that I initially had no interest or experience in. While I had gained significant experience with Raspberry Pi electronic work at my Intel internship before the entire project started, I had never worked on any "practical" applications using the technology yet. Most of my work was at a much lower level, such as writing new SPI device drivers or compatibility testing with $I^{2}C$ peripherals. Moving toward a project that would require me to step out of my C comfort zone into a higher level was shaky at first, but became a crucial part of my development as a software engineer.\\\\
The codebase in Python made up the most of my technical learning on this project. I learned how a web server could be set up in Python and how certain packages, such as Flask, could greatly improve the development process of complex web-centric systems. I learned about optimizing development for mobile devices, since it was the primary way users would be interacting with the Pilight web interface. Additionally, I was able to learn more about the Pi hardware itself and how to manage the limited resources available. Running a traditional web server such as Apache is very intensive on the hardware, but I learned about Flask optimizations that can improve performance. On a side note, I learned about the dangerous power of Git's \textit{rebase} command the hard way. \\\\
My documentation writing skills have greatly improved during this project. Writing endless pages of reports has forced me to improve how I document tasks as the project moves forward, rather than just waiting until the whole project is done. Sean introduced heavy usage of the Github issue tracker, which I initially met with great hostility, but warmed up to over time as it provided an interface to match commits with issue resolutions. My commit messages became clearer and more descriptive with some useful feedback from my group, and I learned how to communicate in an IRC group.\\\\
The project taught me how to abandon my ego and accept that I can make code contributions that can later be deleted as the project progresses. I had always resisted change to my own code, even when it was painfully obvious that it needed to be drastically changed or removed. It was difficult to do at first, as having your code criticized feels like a personal attack sometimes. However, I learned to distance myself from these criticisms and see the changes for the good of the end result.\\\\
Working for three terms taught me that management is fluid, in that there's never a set leader for the entirety of the project. Each one of us switched management duties as our schedules changed over time, to the point where different periods of time had different group members in charge. Our group worked well together and complimented one another's skills well, so we relied on our collective expertise when it was needed for each leg of the project.\\
It was emphasized to be honest in answering the question "If you could do it all over, what would you do differently?", yet I feel that I would not change anything about the project. Sean, Malcolm, and I were a perfect team for this project, where each one of us possessed a unique skillset that made the project work in the end. Along with a flexible client, and a project that was right at our difficulty level, gave us an ideal set of conditions for the project that other groups could only dream of.

\newpage
\section{Code}
The \textit{send\_command(light,action)} function sends the TCP command to the ESP8266 module from the Raspberry Pi. It takes two arguments:
\begin{itemize}
\item \textit{light} = Light object chosen from the database
\item \textit{action} = Boolean for light status ("True" = ON, "False" = OFF)
\end{itemize}
\begin{lstlisting}
def send_command(light, action):
    ip = str(light.device.ipaddr)
    tcp_port = 9999

    command = ""
    for i in range(0,4):
        if light.device.lights[i].port == light.port:
            command += '1' if not action else '0'
        else:
            command += '1' if not light.device.lights[i].status else '0'

    if not debug:
        try:
            sock = socket.socket(socket.AF_INET, socket.SOCK_STREAM)
            sock.connect_ex((ip,tcp_port))
            sock.send(command)
        except socket.error:
            return "Connection refused to device: " + ip

    model.Light.query.filter_by(id=light.id).first().status = action
    model.db.session.commit()
\end{lstlisting}

In the query generator, we have to evaluate the queries from the database to determine what the current state of the light should be. This loop runs every few seconds in a seperate process.

\begin{lstlisting}
# For an explanation of this logic, see here:
        # https://github.com/rettigs/cs-senior-capstone/issues/27#issuecomment-194592403
        if bool(e.status) == previous_rule_state and bool(e.status) != current_rule_state:
            if debug >= 2:
                print "New state:\t{}".format(current_rule_state)

            # Send update to light, or update the group in the database
            if isinstance(e, model.Light):
                send_command(e, current_rule_state)
            else:
                e.status = current_rule_state
        else:
            if debug >= 2:
                print "Not changing state"
\end{lstlisting}

\section{Photos}
\Urlmuskip=0mu plus 1mu\relax
\bibliographystyle{IEEEtran}
\bibliography{end\_report}
\end{document}
