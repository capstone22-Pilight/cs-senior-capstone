The \textit{send\_command(light,action)} function sends the TCP command to the ESP8266 module from the Raspberry Pi. It takes two arguments:
\begin{itemize}
\item \textit{light} = Light object chosen from the database
\item \textit{action} = Boolean for light status ("True" = ON, "False" = OFF)
\end{itemize}
\begin{lstlisting}
def send_command(light, action):
    ip = str(light.device.ipaddr)
    tcp_port = 9999

    command = ""
    for i in range(0,4):
        if light.device.lights[i].port == light.port:
            command += '1' if not action else '0'
        else:
            command += '1' if not light.device.lights[i].status else '0'

    if not debug:
        try:
            sock = socket.socket(socket.AF_INET, socket.SOCK_STREAM)
            sock.connect_ex((ip,tcp_port))
            sock.send(command)
        except socket.error:
            return "Connection refused to device: " + ip

    model.Light.query.filter_by(id=light.id).first().status = action
    model.db.session.commit()
\end{lstlisting}

In the query generator, we have to evaluate the queries from the database to determine what the current state of the light should be. This loop runs every few seconds in a seperate process.

\begin{lstlisting}
# For an explanation of this logic, see here:
        # https://github.com/rettigs/cs-senior-capstone/issues/27#issuecomment-194592403
        if bool(e.status) == previous_rule_state and bool(e.status) != current_rule_state:
            if debug >= 2:
                print "New state:\t{}".format(current_rule_state)

            # Send update to light, or update the group in the database
            if isinstance(e, model.Light):
                send_command(e, current_rule_state)
            else:
                e.status = current_rule_state
        else:
            if debug >= 2:
                print "Not changing state"
\end{lstlisting}
