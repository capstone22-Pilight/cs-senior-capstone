Throughout this project, I learned a lot on the technical side, having not had
much web development experience in the past.  Not only did I get a basic
HTML/CSS refresher, but I also learned basic JavaScript/JQuery and became more
familiar with Flask and Bootstrap.  On the non-technical side, I also was able
to improve my project planning and time management skills; we had to schedule
regular meetings, keep track of all issues, review each other's PRs, and assign
each other tasks to be accomplished by the next meeting.  Using this system, we
were able to make measurable incremental progress throughout the year and
didn't have to scramble to completion in our final weeks.  Overall, I have
found the most important aspect of working in a team to be communication; we
scheduled meetings for at least 2-3 times a week and also kept in constant
contact over IRC in our own channel, allowing us to easily leave messages or
bounce ideas off of each other between meetings.  At the end of each meeting, I
always made sure everyone knew what their action items were and what our
expectations were for the completion of various tasks.  This system made sure
that everyone knew what they needed to do and when, and prevented us from
accidentally duplicating effort.

If I could do it all over, I don't know that I would change much.  For the most
part, our original project goals hadn't changed; we were able to establish
realistic expectations right from the start, and so we were able to
successfully complete our project to our client's satisfaction in the time
allotted.  Overall, the experience was great; the project itself was
interesting, our client was helpful, and we all worked together quite nicely as
an organized group.
