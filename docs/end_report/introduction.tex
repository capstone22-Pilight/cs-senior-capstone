``Not Exactly the Internet of Things for Outdoor Lighting'' (or simply
``PiLight'') is a self-contained home lighting automation system designed to be
powerful and highly customizable, yet inexpensive.  Originally commissioned by
Victor Hsu at Oregon State University for simple outdoor lighting automation,
the project has expanded to include a sophisticated rule and scheduling system
and has the potential to integrate with devices other than lights, such as
sprinkler systems, garage door openers, and more.

While similar products do exist, PiLight attempts to fill a niche between
``overly simplistic'' and ``walled garden''.  Standard wallwart plug timers are
cheap, but their usefulness is limited; if you want your lights to turn on at
sunset, for example, you must manually adjust the timer throughout the year.
If you only want your lights on for the weekends, you're out of luck.
High-tech Internet-connected automation systems are feature-rich, but are often
very expensive and not easily customizable.  Even worse, should the external
Internet services running the show break or shut down, the entire system can be
rendered useless.  PiLight solves these issues by being self-contained; no
external dependencies are required for the system to function normally.
Furthermore, PiLight is built from commodity hardware and open source software,
so the system is not only inexpensive, but also easy to extend and personalize,
for the technically inclined.

PiLight consists of a wireless network of tiny ``client'' computers that each
control up to 4 sets of lights and are controlled by a central ``server''
computer, which automatically sends out commands to the clients when it's time
to turn on or off.  The central node runs a control program that can be easily
accessed via a touch screen, a web browser, or a mobile device, where the user
can locally or remotely control each light individually. The control interface
allows users to easily set ``rules'' for what their lights do and when,
depending on the time of day, day of the week, sun position, and potentially
even triggers such as weather conditions or calendar dates.

A Raspberry Pi with a small touchscreen and Wi-Fi card comprises the central
control unit; it runs its own web server to allow control of lights, as well as
a wireless network that allows both users and client nodes to connect.  The
client nodes are each composed of an ESP8266 Wi-Fi module and a relay, which
automatically connect to the Pi's wireless network when the ``Add Devices''
button is pressed on the web UI.  In a commercial implementation of PiLight,
the client nodes could be integrated into a standard power strip/surge
protector, so users can simply plug their devices in and start using the system
immediately without having to order and wire up the components individually.

The PiLight team is composed of Malcolm Diller, Sean Rettig, and Evan Steele,
who are computer science students at Oregon State University.  Evan Steele's
primary role was creating the Yocto Linux builds for the Raspberry Pi and
writing the ESP8266 firmware, while Malcolm and Sean focused primarily on the
web application, a Flask app with a Bootstrap frontend.  In particular, Sean
developed most of the rule system and advanced settings for lights, while
Malcolm and Evan developed the front page UI, using Javascript to dynamically
update the statuses and names of lights.  Throughout development, we have met
with our client, Victor, to help clarify project requirements, drive
hardware/software design decisions, and provide general guidance to ensure the
project's success.
