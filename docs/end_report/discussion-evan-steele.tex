My experience with the Pilight project gave me the opportunity to merge my existing technical knowledge with an application that I initially had no interest or experience in. While I had gained significant experience with Raspberry Pi electronic work at my Intel internship before the entire project started, I had never worked on any "practical" applications using the technology yet. Most of my work was at a much lower level, such as writing new SPI device drivers or compatibility testing with $I^{2}C$ peripherals. Moving toward a project that would require me to step out of my C comfort zone into a higher level was shaky at first, but became a crucial part of my development as a software engineer.\\\\
The codebase in Python made up the most of my technical learning on this project. I learned how a web server could be set up in Python and how certain packages, such as Flask, could greatly improve the development process of complex web-centric systems. I learned about optimizing development for mobile devices, since it was the primary way users would be interacting with the Pilight web interface. Additionally, I was able to learn more about the Pi hardware itself and how to manage the limited resources available. Running a traditional web server such as Apache is very intensive on the hardware, but I learned about Flask optimizations that can improve performance. On a side note, I learned about the dangerous power of Git's \textit{rebase} command the hard way. \\\\
My documentation writing skills have greatly improved during this project. Writing endless pages of reports has forced me to improve how I document tasks as the project moves forward, rather than just waiting until the whole project is done. Sean introduced heavy usage of the Github issue tracker, which I initially met with great hostility, but warmed up to over time as it provided an interface to match commits with issue resolutions. My commit messages became clearer and more descriptive with some useful feedback from my group, and I learned how to communicate in an IRC group.\\\\
The project taught me how to abandon my ego and accept that I can make code contributions that can later be deleted as the project progresses. I had always resisted change to my own code, even when it was painfully obvious that it needed to be drastically changed or removed. It was difficult to do at first, as having your code criticized feels like a personal attack sometimes. However, I learned to distance myself from these criticisms and see the changes for the good of the end result.\\\\
Working for three terms taught me that management is fluid, in that there's never a set leader for the entirety of the project. Each one of us switched management duties as our schedules changed over time, to the point where different periods of time had different group members in charge. Our group worked well together and complimented one another's skills well, so we relied on our collective expertise when it was needed for each leg of the project.\\
It was emphasized to be honest in answering the question "If you could do it all over, what would you do differently?", yet I feel that I would not change anything about the project. Sean, Malcolm, and I were a perfect team for this project, where each one of us possessed a unique skillset that made the project work in the end. Along with a flexible client, and a project that was right at our difficulty level, gave us an ideal set of conditions for the project that other groups could only dream of.
