There were several technologies we used that many people in our group were not familiar with, but learning them was not too great of a challenge.

\begin{itemize}
	\item Python \\
	While we had all had experience with python before starting on this project, for most of us this was our first large scale project that was almost entirely Python-based. For most questions we had about syntax particulars, we used the standard \href{https://docs.python.org/}{Python Documentation} \cite{python}.
	\item SqlAlchemy \\
	For database managment, we used a Object Relational Mapping (ORM) python library called SqlAlchemy. For information on how to implement it and use it accordingly, we used the \href{http://docs.sqlalchemy.org/}{SqlAlchemy Docs} \cite{sqlalchemy} on the SqlAlchemy Website.
	\item Flask \\
	We decided to use Flask as a framework for our website, which ended up being extremely useful, as it allows for a very seamless integration between the front and back ends of our server. As none of us had used Flask before, the \href{http://flask.pocoo.org/docs/}{Flask Documentation} \cite{flask} was extremely helpful. Another resource that was useful for learning Flask in a more in-depth sense was \href{http://blog.miguelgrinberg.com/post/the-flask-mega-tutorial-part-i-hello-world/}{The Flask Mega-Tutorial} \cite{flaskmega}.
	\item Bootstrap \\
	While some people in the group had used bootstrap before, others were new, and even those who were already familiar with it sometimes needed to look up specifics about how to use it. A resource we found useful was the main \href{http://getbootstrap.com/}{Bootstrap Documentation} \cite{bootstrap}, as well as the \href{http://bootstrap-switch.org/}{Bootstrap Switch} \cite{bootstrapswitch} page for information on how to use the Switches plugin.
	\item Jquery \\
	The Jquery API proved very useful when writing the javascript for our website. As some of us were new to it and others had used it a decent amount, the \href{http://api.jquery.com/}{Jquery Documentation} \cite{jquery} proved quite useful, especially when using the query functionality.
	\item Yocto \\
	Only one of our group members had used the Yocto build environment before, but he was able to provide the advice that we needed to make our project work.
	\item Firmware \\
	We needed to write firmware to make the ESP 8266 module do what we want. This would have been very obscure and hard to do without a reference of some sort, and the one we found useful was a \href{http://github.com/nodemcu/nodemcu-firmware/}{github repository} \cite{firmwarehgithub} that was fairly well organized. We based our firmware off of this code with some slight modifications.
	\item Google \\
	As always when learning new technology, google proved to be an invaluable resource, as the ability to search the web was very useful indeed.
\end{itemize}