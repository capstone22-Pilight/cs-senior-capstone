\documentclass[12pt]{article}

\usepackage{amsmath}
\usepackage{enumitem}
\usepackage[margin=1.5in]{geometry}
\usepackage{hyperref}
\usepackage[utf8]{inputenc}
\usepackage{graphicx}
\usepackage{rotating}

\title{Not exactly the Internet of Things for Outdoor Lighting: Technology Review}
\author{Oregon State University CS Senior Capstone Group 22\\Malcolm Diller, Sean Rettig, Evan Steele\\Client: Victor Hsu}
\begin{document} 

\maketitle

\pagebreak

\section{Client}

Victor Hsu\\
Oregon State University\\
Phone: 541-737-4398\\
Email: hsuv@onid.orst.edu

\section{Problem Statement}

Outdoor lighting seems like a simple problem to solve, but the solutions on the
market today are less than ideal.  The standard transformer/timer combos
available in the big box stores are rudimentary and clunky at best (constantly
needing to be adjusted for the changing sunset and sunrise times), but are
reasonably priced. The new-generation smart apps for home automation are
flexible and fancier, but are quite spendy and lock you into a specific
protocol.  So why not use an open platform running on commodity hardware?  Easy
to use, reasonably priced, and highly customizable--that is our goal. \\

\section{Project Description}

Our system will consist of a wireless network of tiny “client” computers that
each control up to 4 sets of lights and are controlled by a central "server"
computer, which will automatically send out commands to the clients when it's
time to turn on or off.  The central node will run a control program that can
be easily accessed via a touch screen, a web browser, or a mobile device, where
the user can locally or remotely control each light individually.  Want your
lights to turn on at sunset and then dim gradually as the sun rises?  Simple.
Want your lights to flash when you're throwing a party?  Just press a button.
The control interface will allow users to easily set "rules" for what their
lights do and when, depending on the time of day, the sun/moon position, and
potentially even triggers such as weather conditions or calendar dates.  This
system will be easily extensible to potentially control other devices as well,
such as garage doors, sound systems, and more. \\

\section{Pieces}

\begin{enumerate}
    \item Server OS for the PI
        \begin{enumerate}
            \item Raspbian \\
                Custom built for the PI, this operating system would provide many important features, and be a solid foundation to run on.
            \item Yocto build of Linux \\
                We will not need all of the fancy features that Raspbian offers, and one way we can strip our OS down to the features we need is by building a Linux system using Yocto.
            \item something else?
        \end{enumerate}
        We have decided to use a yocto build of linux because (reasons)
    \item Server Program Implementation
        \begin{enumerate}
            \item Python / MicroPython
            \item C
            \item Java?
        \end{enumerate}
        We have decided to use Python and MicroPython because (reasons)
    \item Web Site Implementation
        \begin{enumerate}
            \item JavaScript
            \item HTML
            \item PHP
        \end{enumerate}
        We have decided to use (html \& php?) because (reasons)
    \item Server/Client Communication
        \begin{enumerate}
            \item Sockets
            \item Technology2
            \item Technology3
        \end{enumerate}
        We have decided to use Sockets because (reasons)
\end{enumerate}
    
\end{document}
