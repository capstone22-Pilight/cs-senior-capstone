\documentclass[oneside,openright]{scrreprt}
\usepackage{listings}
\usepackage{underscore}
\usepackage[bookmarks=true]{hyperref}
\usepackage{atbegshi}% http://ctan.org/pkg/atbegshi
\AtBeginDocument{\AtBeginShipoutNext{\AtBeginShipoutDiscard}}
\hypersetup{
	bookmarks=false,    % show bookmarks bar?
		pdftitle={Software Requirement Specification},    % title
		pdfauthor={Yiannis Lazarides},                     % author
		pdfsubject={TeX and LaTeX},                        % subject of the document
		pdfkeywords={TeX, LaTeX, graphics, images}, % list of keywords
		colorlinks=true,       % false: boxed links; true: colored links
		linkcolor=blue,       % color of internal links
		citecolor=black,       % color of links to bibliography
		filecolor=black,        % color of file links
		urlcolor=purple,        % color of external links
		linktoc=page            % only page is linked
}%
\def\myversion{1.0 }
\title{
	\flushright
		\rule{16cm}{5pt}\vskip1cm
		\Huge{SOFTWARE REQUIREMENTS}\\
		\vspace{2cm}
	for\\
		\vspace{2cm}
	Not Exactly the Internet of Things for Outdoor Lighting\\
		\vspace{2cm}
	\LARGE{Conceptual Release:}
	\vspace{2cm}
	\LARGE{Subject to change\\}
	\vspace{2cm}
	Prepared by Evan Steele, Sean Rettig, and Malcolm Diller\\
		\vfill
		\rule{16cm}{5pt}
}

\date{}
\usepackage{hyperref}
\begin{document}
\maketitle
\tableofcontents
\newpage

\section{Overview}

In the twenty-first century, your home should be smart and responsive. An
automated home should be able to control appliances such as lights through
single-board computer units that have Internet connectivity. Many commercial
options exist, such as those from Nest Labs, but such devices are prohibitively
expensive for most consumers, especially for simple tasks such as lighting
automation. This project aims to provide an low-cost alternative for simple
home automation, for devices like lights and garage door openers. Using a
low-cost embedded computer such as the Raspberry Pi and cheap wireless
communication modules such as the ESP8266, this project will provide a cheap,
easy-to-install home automation system.

The project is called "Not Exactly the Internet-of-Things" because it's not
designed as a typical IoT product. The network is entirely local, not worrying
about an external connection to the Wide Area Network. Some additional
components such as consulting an external API could be considered, but that's
an addon to the finished product. As it is made as a home application, it
is readily accessible through smartphones and web interfaces. With a clean user
interface that enables both manual control and automation profiles, the
application will require a minimal learning curve and be simple to set up.  

\section{Introduction}

\section{Purpose}

\section{Project Scope and Product Features}

\section{References}

\section{Overall Description}

\section{Product Perspective}

The final deliverable of a commercial version of this product would likely include the following:

\begin{itemize}
    \item A central control unit consisting of the Raspberry Pi, its power
cable, the touchscreen module, a case, and an SD card with the server software
preloaded onto it.
    \item One or more plug units which would likely resemble a power strip,
with each plug being individually controllable by the system.
\end{itemize}

To set up the system, the user simply plugs both the control unit and the plug
unit(s) into wall power.  The plug units will preconfigured to connect to the
control unit.  The user can then plug devices into the plug units, such as
lamps, and the display on the control unit will update itself automatically
with a list of lights.  The lights can have either numbered (e.g. "Zone 1,
Light 1") or randomized but memorable names by default (e.g. "Blue Koala")
which can be later edited through the web interface.  To access the web
interface, the user connects to the wifi network provided by the control unit
using the credentials included with the manual, or perhaps printed on the
control unit itself.  They then open a web browser and connect to a specific
URL, provided to the user with the wifi password.  From the web interface, all
features are accessble, such as renaming lights, creating light groups, and
applying rules for when to toggle lights.

\section{Testing}

To test the functionality and correctness of our system, we will test the various components both individually and when integrated with other components:

\begin{itemize}
    \item We plan to test the wireless communication capabilities of the system
by attempting to send packets between the control unit and plug units at
varying distances and in varying environments, including open and walled areas.
Ideally, the range should be great enough to fully cover the average American
house/apartment.  Once this is complete, it will likely not need to be changed
or extended much, reducing the need for regression testing.
    \item We plan to test the actual light toggling (as performed by the plug
units) manually.  Once this is complete, it will likely not need to be changed
or extended much, reducing the need for regression testing.
    \item We plan to test the functionality of the touchscreen interface
through manual human interaction testing.  Once this is complete, it will
likely not need to be changed or extended much, reducing the need for
regression testing.
    \item We plan to test the functionality of the web interface through a
combination of manual human interaction testing, automated API-driven testing,
and automated GUI testing (using a platform such as Selenium).  The plug units
can send back acknowledgement packets to confirm what actions they took in
response to the tests, allowing nearly complete integration tests to be
performed automatically when changes are made to the control program and web
interface.  These automated tests will help prevent regressions as new features
are developed and save time in ensuring the correctness of the system.
\end{itemize}

\section{Timeline}
% Should probably change table to look better later
\begin{tabular}{ | l | p{0.8\linewidth} | }
   Device control & \\ \hline

   Server can boot with GUI and touchscreen support & Thu 11/19/15 \\ \hline
   Server starts control program on power on & Mon 11/23/15 \\ \hline
   Clients are loaded with client control program & Thu 11/12/15 \\ \hline
   Clients discover connected relays/lights & Thu 11/19/15 \\ \hline
   Clients can toggle lights individually & Tue 11/24/15 \\ \hline
   Lights can be toggled over time & Fri 11/27/15 \\ \hline
   Server acts as wireless access point for clients & Mon 11/30/15 \\ \hline
   Clients can be paired with servers & Fri 12/4/15 \\ \hline
   Clients tell server which lights are available & Tue 12/8/15 \\ \hline
   Server can send light instructions to clients & Thu 12/10/15 \\ \hline
   Clients can receive light instructions from server & Mon 12/14/15 \\ \hline
   Clients can perform light instructions from server & Wed 12/16/15 \\ \hline
   
   User Interface & \\ \hline
   
   Server serves web site with control interface for user & Mon 11/30/15 \\ \hline
   Accessible via the system’s built-in touchscreen & Mon 12/7/15 \\ \hline
   Accessible to other devices like phones & Thu 12/3/15 \\ \hline
   Displays list of connected clients & Mon 12/7/15 \\ \hline
   Can nickname clients & Tue 12/8/15 \\ \hline
   Displays list of connected lights & Wed 12/9/15 \\ \hline
   Can nickname lights & Thu 12/10/15 \\ \hline
   Buttons to toggle lights individually & Thu 12/10/15 \\ \hline
   Sliders to change light intensity & Fri 12/11/15 \\ \hline
   Can put lights into groups & Mon 12/14/15 \\ \hline
   Displays list of groups & Tue 12/15/15 \\ \hline
   Can nickname groups & Wed 12/16/15 \\ \hline
   Groups can be nested & Wed 12/23/15 \\ \hline
   Can toggle lights/groups based on rules & Fri 1/1/16 \\ \hline
   
   Rules & \\ \hline
   
   Lights can be toggled manually, overriding any rules & Tue 12/15/15 \\ \hline
   Rules can be temporarily disabled & Wed 1/6/16 \\ \hline
   Lights can be toggled based on time of day & Tue 1/12/16 \\ \hline
   Lights can be toggled based on day of week & Thu 1/14/16 \\ \hline
   Lights can be toggled based on day of month & Thu 1/14/16 \\ \hline
   Lights can be toggled based on day of year & Thu 1/14/16 \\ \hline
   Lights can be toggled based on specific dates & Thu 1/14/16 \\ \hline
   Lights can be toggled based on sunrise/sunset times & Fri 1/22/16 \\ \hline
   Multiple rules per light/group & Fri 1/15/16 \\ \hline
   Lights can be toggled based on parent group & Tue 1/5/16 \\ \hline
   Rules can toggle early/late by random time & Tue 1/5/16 \\ \hline
   Lights can toggle early/late randomly within groups & Wed 1/6/16 \\ \hline
   Lights can gradually toggle over time & Thu 1/7/16 \\ \hline
\end{tabular}

\section{User Classes and Characteristics}

% add other chapters and sections to suit

\end{document}
