\documentclass[oneside,openright]{scrreprt}
\usepackage{listings}
\usepackage{underscore}
\usepackage[bookmarks=true]{hyperref}
\usepackage{atbegshi}% http://ctan.org/pkg/atbegshi
\AtBeginDocument{\AtBeginShipoutNext{\AtBeginShipoutDiscard}}
\hypersetup{
	bookmarks=false,    % show bookmarks bar?
		pdftitle={Software Requirement Specification},    % title
		pdfauthor={Yiannis Lazarides},                     % author
		pdfsubject={TeX and LaTeX},                        % subject of the document
		pdfkeywords={TeX, LaTeX, graphics, images}, % list of keywords
		colorlinks=true,       % false: boxed links; true: colored links
		linkcolor=blue,       % color of internal links
		citecolor=black,       % color of links to bibliography
		filecolor=black,        % color of file links
		urlcolor=purple,        % color of external links
		linktoc=page            % only page is linked
}%
\def\myversion{1.0 }
\title{
	\flushright
		\rule{16cm}{5pt}\vskip1cm
		\Huge{SOFTWARE REQUIREMENTS}\\
		\vspace{2cm}
	for\\
		\vspace{2cm}
	Not Exactly the Internet of Things for Outdoor Lighting\\
		\vspace{2cm}
	\LARGE{Conceptual Release:}
	\vspace{2cm}
	\LARGE{Subject to change\\}
	\vspace{2cm}
	Prepared by Evan Steele, Sean Rettig, and Malcom McDiller\\
		\vfill
		\rule{16cm}{5pt}
}

\date{}
\usepackage{hyperref}
\begin{document}
\maketitle
\tableofcontents
\newpage
\section{Overview}
In the twenty-first century, your home should be smart and responsive. An automated home should be able to control appliances such as lights through single-board computer units that have Internet connectivity. Many commercial options exist such as those from Nest Labs, but such devices are prohibitively expensive for most consumers, especially for simple tasks such as lighting automation. This project aims to provide an low-cost alternative for simple home automation, such as for devices that are exclusively binary like lights and simple heating elements. Using a low-cost embedded computer such as the Raspberry Pi and cheap wireless communication modules such as the ESP8266, this project will provide a cheap, easy-to-install home automation system.\\
\\The project is called "Not Exactly the Internet-of-Things" because it's not designed as a typical IoT product. The network is entirely local, not worrying about an external connection to the Wide Area Network. Some additional components such as consulting an external API could be considered, but that's an addon to the finished product. As it is made as a user home application, it is readily accessible through smartphones and web interfaces. With a clean user interface that enables both manual control and automation profiles, the application will require a minimal learning curve and be simple to set up.  
\section{Introduction}
\section{Purpose}
\section{Project Scope and Product Features}
\section{References}
\section{Overall Description}
\section{Product Perspective}
\section{User Classes and Characteristics}
% add other chapters and sections to suit
\end{document}
