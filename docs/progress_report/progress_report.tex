\documentclass[letterpaper,10pt]{article}
\usepackage[margin=0.75in]{geometry}
\usepackage{graphicx}

\title{Not Exactly the Internet of Things for Outdoor Lighting:\\Progress Report}
\author{Oregon State University CS Senior Capstone Group 22\\
Malcolm Diller, Sean Rettig, Evan Steele\\
Client: Victor Hsu}
\date{\today, Fall Term}

\begin{document}

\maketitle

\section{Abstract}

Our group has made serious progress in the project to turn the Raspberry Pi and
ESP8266 Wifi module into a "Not exactly IoT" device for controlling binary
devices wirelessly. From project assignment to week five, our group produced a
proof-of-concept for the final project, and planning the final software
approach in the meantime. Despite all the writing requirements this term, we
made excellent progress.

Throughout the term, our team moved between early prototyping and detailed
documentation for the project. With regular communication through email and IRC
our group kept coordinated throughout the term, publishing progress reports to
the Sharepoint site and code samples to our Github page. We will discuss our
weekly progress and describe how the prototype and documentation unfolded over
the course of the first term of the capstone project. This should provide an
assessment on where our groups stands in preparedness to move forward with
implementation next term.

\pagebreak

\section{Weekly Breakdown}

\subsection{Week 3}

Our first week of the capstone project ``Not exactly the Internet of Things for
Outdoor Lighting'' was relatively uneventful, as we spent most of the time
coordinating group meetings.  At our first group meeting, we brainstormed a
list of clarifying questions to ask our client, Victor, about the project,
including:

\begin{itemize}
    \item How is the system going to be controlled by phones?  Is a dedicated
        app necessary, or can we provide a mobile-friendly website?\\ Answer: A
        mobile-friendly website is acceptable.
    \item Are we just going to be controlling lights, or are we going to expand
        to other devices if we have time (e.g. sprinkler system, music players,
        garage door openers, alarm systems)?\\
        Answer: The control of other devices is currently outside the scope of
        this project, but can be a stretch goal.
    \item Will the code for this project be open source?\\
        Answer: Yes!  The entire project is hosted on Github.com in a public
        repository.
    \item What hardware will we have access to?\\
        Answer: In addition to the hardware mentioned in the client's original
        project proposal, Victor agreed to order a 2.8" TFT Display with
        Resistive Touchscreen <http://www.adafruit.com/products/1774> for use
        with the Raspberry Pi in the control unit.
\end{itemize}

In addition, we were able to finish our Problem Statement document, which
largely involved expanding upon our client's original project proposal with
additional clarification, focus, technical details, and performance metrics.

\subsection{Week 4}
 
Having not received the hardware from our client yet, there was not much we
could do on the physical side.  We did, however, begin work on our Requirements
Document, listing out several core requirements, as well as some stretch goals.
Much of this work was simply listing out each individual piece of the system,
all the while filling in assumptions with explicit requirements and providing
examples to ensure the requirements cannot be misconstrued.  For some of the
aspects that we not yet sure of, we injected a little of our own vision,
knowing that our client would be able to make the final decisions on details
later.  For organizational purposes, we decided to break up the requirements
into three categories:

\begin{enumerate}
    \item Device control
    \item User interface
    \item Rules
\end{enumerate}

The ``device control'' section includes the bare essentials for booting each
device, running the control programs, and ensuring that the devices are able to
communicate with each other wirelessly.  The ``user interface'' section
includes requirements regarding the user's experience and access through the
touchscreen and web site interfaces, including how they can view, name, group,
toggle, and apply rules to lights.  The ``rules'' section, then, describes the
rules that will be available for users to control their lights with.  These
include time-based rules (e.g. turning on at 8pm and off at 6am every weekday)
and of course, the crowning feature, rules for toggling based on the sunset and
sunrise, obviating the need to manually adjust the system when the sunset and
sunrise times change with the seasons.

We also included a ``stretch goals'' category following the same 3-section
layout as described above, where we filled in interesting ideas that we and our
client had, but were not absolutely necessary for the proper functioning of the
system and would only be implemented after the core requirements were complete,
if time allowed.  In the device control section, stretch goals included
potentially expanding the system to control devices other than lights, such as
garage door openers, music players, holiday decorations, and sprinklers.  In
the user interface section, we considered additional usability features, such
as allowing the web interface to be accessible over the Internet (for remote
control) or even external "cloud" hosting for the web interface to improve
uptime and obviate the need for the user to port forward on their router.  The
rules section included more complex rules, such as toggling based on
weather/celestial conditions (which might require Internet access) or various
attachable sensors, such as motion sensors, light sensors, moisture sensors,
and sound sensors.  These types of rules would require significant additional
hardware, but would allow the system to, for example, turn on lights when you
pull into the driveway.  The usefulness of these rules would also greatly
increase if combined with other stretch goals, like sprinkler system control,
as the system could then be configured to stop sprinkling if the ground is
already wet (e.g. due to rain) or to pause temporarily if someone is walking
by.

\subsection{Week 5}

This week, we completed our requirements document and began researching
the hardware once we had it in our hands. We picked up the Raspberry Pi
and TFTLCD display from Victor and booted the device with the \textit{fbtft} 
module for the device. This week was otherwise uneventful with everyone working
on other projects and midterms, but there was still work done with regards to 
group logistics and future planning. When we aren't so busy next week, we will
continue our proof of concept project to get the firmware and relay online.

\subsection{Week 6}
 
Most of what we did on week 6 of our project involved trying to get a
functioning proof of concept for our project, which turned out to be filled
with more problems than anticipated. We began to investigate software packages
and technologies that would help us meet the requirements outlined on our
document. The main problem that we ran into this wee was with the wireless
networking that we were planning on for the Pi.

We tested the devices with various wireless networking methods including ad-hoc
and in access-point mode. While testing, we discovered that the ad-hoc system
was not working as expected. This was due to the USB WiFi adapters we were
using not being compatible with an ad-hoc setup. This was especially
problematic because our main plan was to use ad-hoc for networking. After
further investigation, it became clear that we would have to reevaluate our
network implementation, as there would be no way to work around the issue and
still use ad-hoc. We are now just using the Pi as a wireless access point. This
solves the problems caused by the compatibility issues with ad-hoc.

In addition to working on the proof of concept, we also made headway on some of
the documents that we had due soon. As we continued to develop the Technical
Requirements document, we asked the TA to see if he had any comments on it. His
advice was to elaborate more on the requirements we already had and make them
as specific as possible. Following his advice, we expanded on our current
requirements by adding more detail and specifying exactly what we meant. We
also refined our elevator pitch to make it a better fit for the intended
audience.

\subsection{Week 7}  

After much delay, sweat and tears, we finally have a functional proof 
of concept to demo and show off. The program is super simple but 
accomplishes the bare minimum of what we would expect the program to do.
Sending a basic TCP command to the ESP8266 module will cycle the lights 
through the device's Lua firmware, and will in turn flip the relays out of
the board's GPIO once we wire it up too. The TCP commands are sent by the 
Raspberry Pi using a boot script on lxTerminal, using an execution sequence
that looks like this:

\begin{enumerate}
    \item Pi boots into the TFTLCD module, screen initialized
    \item Pi loads lxTerminal, runs startup script in init.d
    \item inid.d script connects to the ESP8266 module, gets an IP
    \item After the connection is made, the inid.d script starts a Python
        script
    \item The Python script cycles through TCP commands and send them to the
        device
\end{enumerate}

For our next step, we need to make sure that the Pi can reliably perform the
discovery operation (the IP is hard-coded right now) and that the ESP8266
module can run as a client in the future, connecting to the Pi to take
advantage of the Pi's DHCP server and to simplify future connections.\\

\includegraphics[scale=0.1]{circuit.jpg}

\subsection{Week 8}
 
Progress since last week: 
 
Our proof of concept is now portable, and booting the pi connects it to the esp and cycles the lights on the relay. It has also been shown to the TA. 
Completed the expo poster 
Re-Wrote the elevator pitch 
Problems: 
No real problems this week 
Plans for next week: 
Continue to develop the proof of concept, possibly by adding actual lights to the relay instead of the LED's. 
Present the elevator pitch 

\subsection{Week 9}
 
On Week 9 we continued to work on testing out our proof of concept. We managed
to connect actual lights to our relay and verified that we can control them by
toggling them on and off. This is a good step toward getting a fully working
proof of concept working as it means we have most of the main parts done, and
mainly need to start working on the UI of the touchscreen and web page.

We decided that we are most likely going to use Flask to implement the website.
Flask is a micro-framework for Python that is designed for developing websites.
The main reason we are planning on using it is that we are using Python for
many of the other parts of the project, and so it would be easy to integrate
this and also would make sense in terms of keeping the code consistent. To
prepare for when we are ready to start implementing that, we have begun
learning more about how Flask works, and how we might be able to use it to
create our our website.

We also presented our elevator pitch on this week, which went decently well.
There were no documents to turn in this week, and there was only class on
Tuesday so there is not much else to report. Next week we planned to iron out
the design document and work on the progress report. 

\subsection{Week 10}

The final week of the capstone project was largely uneventful as our group
began working on other final projects and studying for exams. However, we did
get some work in with regards to some project details and the work on the
design document.  This week we took our document to the TA to get some feedback
regarding the design elements we focused on for the paper, and he was pleased
with the overall result after some vocabulary changes.  The hardware project
itself didn't get much direct attention this week, but we did wonder about the
usability of the final result, from a user perspective.  How easy-to-use should
set-up be? Should the user have to worry about port forwarding on the home
network?  According to Victor, our project sponsor, there was no need to design
an overly-simplified option.  We will continue to investigate more networking
options over break.

\begin{itemize}
    \item Completed: Design Document\\
        The design document was reviewed by our group and a TA and turned in.
        we also had our weekly meeting with the TA at this time.  
    \item How complicated does the user-setup portion of the project need to be
        and are there restrictions on what they should be able to do?\\
        Answer: As of now, having to reconnect to the device to reconfigure it
        is acceptable. Victor imagines that we won't have to connect so often
        that investing time into developing a complex networking option is not
        necessary.
    \item What will the project do over break?\\
        Answer: Evan will be taking the hardware home to use in his network
        testing bench, which is pretty much just a dozen microcomputers plugged
        into an old Cisco 2960x networking together. 
\end{itemize}

Over break, Evan will be experimenting with some networking strategies and
working with the web serve to flip the relay. Sean suggested that Flask would
provide a good framework, but as of now he's the only one with experience in
it, so Malcolm and Evan will need to study it. An ideal goal for the start of
Winter term would be having a working web interface that can flip the relay,
built with the Yocto build server. Even better if we have the interface up on
the TFTLCD screen, but we may save that until we get back together.

\end{document}
