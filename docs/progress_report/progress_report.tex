\documentclass[letterpaper,10pt]{article}
\usepackage[margin=0.75in]{geometry}

\title{Not Exactly the Internet of Things for Outdoor Lighting:\\Progress Report}
\author{Oregon State University CS Senior Capstone Group 22\\Malcolm Diller, Sean Rettig, Evan Steele\\Client: Victor Hsu}
\date{\today, Fall Term}

\begin{document}

\maketitle

\section{Abstract}

abstract goes here

\pagebreak

\section{Progress}

% sean
\subsection{Week 3}

Our first week of the capstone project ``Not exactly the Internet of Things for
Outdoor Lighting'' was relatively uneventful, as we spent most of the time
coordinating group meetings.  At our first group meeting, we brainstormed a
list of clarifying questions to ask our client, Victor, about the project,
including:

\begin{itemize}
    \item How is the system going to be controlled by phones?  Is a dedicated
        app necessary, or can we provide a mobile-friendly website?\\ Answer: A
        mobile-friendly website is acceptable.
    \item Are we just going to be controlling lights, or are we going to expand
        to other devices if we have time (e.g. sprinkler system, music players,
        garage door openers, alarm systems)?\\
        Answer: The control of other devices is currently outside the scope of
        this project, but can be a stretch goal.
    \item Will the code for this project be open source?\\
        Answer: Yes!  The entire project is hosted on Github.com in a public
        repository.
    \item What hardware will we have access to?\\
        Answer: In addition to the hardware mentioned in the client's original
        project proposal, Victor agreed to order a 2.8" TFT Display with
        Resistive Touchscreen <http://www.adafruit.com/products/1774> for use
        with the Raspberry Pi in the control unit.
\end{itemize}

In addition, we were able to finish our Problem Statement document, which
largely involved expanding upon our client's original project proposal with
additional clarification, focus, technical details, and performance metrics.

% sean
\subsection{Week 4}
 
Progress since last week: 
We have started on our requirements document and listed the required and optional requirements. 
Problems: 
No problems this week. 
Plans for the coming week: 
Meet with our client to review our draft of the requirements document and finalize it. 
Start to work with the hardware as soon as we get it. This will probably consist of booting the Raspberry Pi with an operating system and compiling MicroPython on the Wifi Modules. 
If we get far enough, work on communication between the Pi and the controllers. 

\subsection{Week 5}
 
Progress since last week: 
We have finished the requirements document after having our client review it. 
We received our Raspberry Pi and touchscreen hardware. 
Problems: 
No problems this week 
Plans for the coming week: 
Begin working on the Pi starting with an operating system and then branching out to communication between the devices and touchscreen capabilities. 

% Malcolm
\subsection{Week 6}
 
Most of what we did on week 6 of our project involved trying to get a functioning proof of concept for our project, which turned out to be filled with more problems than anticipated. We began to investigate software packages and technologies that would help us meet the requirements outlined on our document. The main problem that we ran into this wee was with the wireless networking that we were planning on for the Pi.

We tested the devices with various wireless networking methods including ad-hoc and in access-point mode. While testing, we discovered that the ad-hoc system was not working as expected. This was due to the USB WiFi adapters we were using not being compatible with an ad-hoc setup. This was especially problematic because our main plan was to use ad-hoc for networking. After further investigation, it became clear that we would have to reevaluate our network implementation, as there would be no way to work around the issue and still use ad-hoc. We are now just using the Pi as a wireless access point. This solves the problems caused by the compatibility issues with ad-hoc.

In addition to working on the proof of concept, we also made headway on some of the documents that we had due soon. As we continued to develop the Technical Requirements document, we asked the TA to see if he had any comments on it. His advice was to elaborate more on the requirements we already had and make them as specific as possible. Following his advice, we expanded on our current requirements by adding more detail and specifying exactly what we meant. We also refined our elevator pitch to make it a better fit for the intended audience.

% Evan
\subsection{Week 7}

Progress since last week: 
We now have a working proof of concept where the Pi is able to wirelessly communicate to the ESPs and cause each of the connections on the relays to be activated individually. 
Completed technology review and revision of requirements document. 
Started on expo poster. 
Problems: 
No real problems other than having to revise the requirements document to be more detailed and fine-grained. 
Plans for next week: 
Finish poster 
Revise elevator pitch 
Demonstrate proof of concept to TA 
Possibly attach actual lights to the relay rather than just using the relay's LEDs to see which switches are enabled. 

Week 7 saw our first bit of prototyping that ended in a successful test. The Raspberry Pi can now send TCP commands to the ESP8266 module. A test program cycles through the command bits and all the lights on the relay turn on in sequence. We continued to develop the prototype  

Our first week of the capstone project ``Not exactly the Internet of Things for
Outdoor Lighting'' was relatively uneventful, as we spent most of the time
coordinating group meetings.  At our first group meeting, we brainstormed a
list of clarifying questions to ask our client, Victor, about the project,
including:

\begin{itemize}
    \item How is the system going to be controlled by phones?  Is a dedicated
        app necessary, or can we provide a mobile-friendly website?\\ Answer: A
        mobile-friendly website is acceptable.
    \item Are we just going to be controlling lights, or are we going to expand
        to other devices if we have time (e.g. sprinkler system, music players,
        garage door openers, alarm systems)?\\
        Answer: The control of other devices is currently outside the scope of
        this project, but can be a stretch goal.
    \item Will the code for this project be open source?\\
        Answer: Yes!  The entire project is hosted on Github.com in a public
        repository.
    \item What hardware will we have access to?\\
        Answer: In addition to the hardware mentioned in the client's original
        project proposal, Victor agreed to order a 2.8" TFT Display with
        Resistive Touchscreen <http://www.adafruit.com/products/1774> for use
        with the Raspberry Pi in the control unit.
\end{itemize}

In addition, we were able to finish our Problem Statement document, which
largely involved expanding upon our client's original project proposal with
additional clarification, focus, technical details, and performance metrics.

% sean
% Evan
\subsection{Week 8}
 
Progress since last week: 
 
Our proof of concept is now portable, and booting the pi connects it to the esp and cycles the lights on the relay. It has also been shown to the TA. 
Completed the expo poster 
Re-Wrote the elevator pitch 
Problems: 
No real problems this week 
Plans for next week: 
Continue to develop the proof of concept, possibly by adding actual lights to the relay instead of the LED's. 
Present the elevator pitch 

% Malcolm
\subsection{Week 9}
 
On Week 9 we continued to work on testing out our proof of concept. We managed to connect actual lights to our relay and verified that we can control them by toggling them on and off. This is a good step toward getting a fully working proof of concept working as it means we have most of the main parts done, and mainly need to start working on the UI of the touchscreen and web page.

We decided that we are most likely going to use Flask to implement the website. Flask is a micro-framework for Python that is designed for developing websites. The main reason we are planning on using it is that we are using Python for many of the other parts of the project, and so it would be easy to integrate this and also would make sense in terms of keeping the code consistent. To prepare for when we are ready to start implementing that, we have begun learning more about how Flask works, and how we might be able to use it to create our our website.

We also presented our elevator pitch on this week, which went decently well. There were no documents to turn in this week, and there was only class on Tuesday so there is not much else to report. Next week we planned to iron out the design document and work on the progress report. 

% Evan
\subsection{Week 10}

\end{document}
