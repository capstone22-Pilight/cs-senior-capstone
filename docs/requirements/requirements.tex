\documentclass[12pt]{article}

\usepackage{amsmath}
\usepackage{enumitem}
\usepackage{hyperref}
\usepackage[utf8]{inputenc}
\usepackage{graphicx}
\usepackage{rotating}

\usepackage{geometry}
%for 0.75" margins
\geometry{textheight=9.5in, textwidth=7in}
%for 1" margins
%\geometry{textheight=9in, textwidth=6.5in}

\title{Not exactly the Internet of Things for Outdoor Lighting: Requirements Document}
\author{Oregon State University CS Senior Capstone Group 22\\Malcolm Diller, Sean Rettig, Evan Steele\\Client: Victor Hsu}
\begin{document} 

\maketitle

\pagebreak

\tableofcontents

\pagebreak

\section{Team}

\subsection{Team Name}

Cupcake Warriors

\subsection{Team Members}

\begin{tabular}{ l l } 
    Malcolm Diller & dillerm@oregonstate.edu\\ 
    Sean Rettig & rettigs@oregonstate.edu\\
    Evan Steele & steelee@oregonstate.edu\\ \end{tabular}

\subsection{Client}

Victor Hsu\\
Oregon State University\\
Phone: 541-737-4398\\
Email: hsuv@onid.orst.edu

\section{Introduction}

\subsection{Problem Statement}

Outdoor lighting seems like a simple problem to solve, but the solutions on the
market today are less than ideal.  The standard transformer/timer combos
available in the big box stores are rudimentary and clunky at best (constantly
needing to be adjusted for the changing sunset and sunrise times), but are
reasonably priced. The new-generation smart apps for home automation are
flexible and fancier, but are quite spendy and lock you into a specific
protocol.  So why not use an open platform running on commodity hardware?  Easy
to use, reasonably priced, and highly customizable--that is our goal.

\subsection{Project Description}

Our system will consist of a wireless network of tiny ``client'' computers that
each control up to 4 sets of lights and are controlled by a central ``server''
computer, which will automatically send out commands to the clients when it's
time to turn on or off.  The central node will run a control program that can
be easily accessed via a touch screen, a web browser, or a mobile device, where
the user can locally or remotely control each light individually.  Want your
lights to turn on at sunset and then dim gradually as the sun rises?  Simple.
Want your lights to flash when you're throwing a party?  Just press a button.
The control interface will allow users to easily set ``rules'' for what their
lights do and when, depending on the time of day, the sun/moon position, and
potentially even triggers such as weather conditions or calendar dates.  This
system will be easily extensible to potentially control other devices as well,
such as garage doors, sound systems, and more.

\subsection{Design}

Specifically, we plan to use a Raspberry Pi running Linux as the central server
computer.  The Raspberry Pi will run a web server program that will allow
nearby devices (such as laptops, phones, or tablets) to control it over a
wireless network, or if connected to the home's internet connection, from
practically anywhere in the world.  The Pi will also have a small touchscreen
connected to it with the website open in a browser, so the user has a dedicated
interface for the device.  The Pi will connect wirelessly to small wifi-enabled
microcontrollers placed throughout the home, each of which are connected to up
to 4 relays to control 4 sets of lights.

\section{Requirements}

\subsection{Critical Requirements}

The following requirements are critical to the basic functionality of the
system and must be functional prior to the expo in spring term.

\begin{enumerate}
    \item Device control
        \begin{enumerate}
            \item Server is loaded with Linux and is able to boot up to a
                graphical interface that can be viewed and controlled by the
                device's touchscreen.
            \item Server is able to start the server control program
                automatically when powered on.
            \item Clients are loaded with the client control program, which
                runs automatically when the device is powered on.
            \item Client control program is able to automatically discover
                relays/lights that are connected to it.
            \item Client control program can control the relays to power the
                lights on/off individually.
            \item Lights can be toggled over a user-specified period of time,
                e.g. a light can gradually turn on and grow brighter over the
                course of 30 seconds.
            \item Server is able to act as a wireless access point that each
                client can connect to.
            \item Clients can be paired with a server by putting the server
                into ``pairing mode'' (by pressing a button either on the user
                interface or on the Pi itself) that will last for a short time
                (just a few seconds). The server will then wirelessly broadcast
                UDP packets that nearby unpaired clients will see and use to
                connect to the server. This system will allow multiple servers
                to be used in close proximity with separate sets of clients.
            \item Clients are able to communicate to the server which lights
                are available for control.
            \item Server is able to send light toggle instructions wirelessly
                to clients.
            \item Clients are able to read light toggle instructions wirelessly
                from the server.
            \item Clients are able to perform light toggle instructions
                received from the server.
        \end{enumerate}

    \item User interface
        \begin{enumerate}
            \item Main control program on server serves a web site with a
                control interface for the user.
            \item The web interface is accessible via the system's built-in
                touchscreen.
            \item The web interface is accessible to other devices like laptops
                and phones via a local wireless network.
            \item The web interface displays a list of connected clients.
            \item The web interface allows users to give each client a
                ``nickname'' for easy identification.
            \item The web interface displays a list of connected lights.
            \item The web interface allows users to give each light a
                ``nickname'' for easy identification.
            \item The web interface contains buttons for each light that can be
                used to toggle lights individually.
            \item The web interface contains sliders for each light that can be
                used to change the intensity of lights individually.
            \item The web interface contains the ability to put lights into
                ``groups'' that can be controlled together all at once.
            \item The web interface displays a list of light groups.
            \item The web interface allows users to give each light group a
                ``nickname'' for easy identification.
            \item Groups can also be nested into other groups to create
                hierarchies of lights.  For example, there can be two groups
                called ``Front Porch'' and ``Back Porch'' that control the
                front and back porch lights, respectively.  Both groups can
                then be inside another group called ``House''.  If the
                ``House'' group is toggled on, then both the front and back
                porch lights will be toggled on.  If just the front porch
                lights are then toggled off, the back lights will stay on.
            \item The web interface provides the user with options to toggle
                lights/groups based on rules, as described in the below
                ``Rules'' section.
        \end{enumerate}

    \item Rules
        \begin{enumerate}
            \item Lights can be toggled manually, overriding any rules (as
                described in the previous section). The manual setting will
                stay in effect until another rule is triggered. For example, if
                your lights are set to turn off every morning at 6am, but you
                manually turn them on at 7am, they will stay on all day until
                the next morning at 6am.
            \item Rules can be toggled manually to temporarily disable them.
                For example, if you have your lights set to turn on every night
                at 8pm, but are going on vacation for a week and don't need
                them, you can turn that rule off and it will stop triggering
                until you turn it on again, keeping the lights off until after
                you get back. This way, the rule doesn't need to be deleted and
                completely recreated from scratch.
            \item Lights can be toggled based on time of day (e.g. ``turn on at
                8pm and turn off at 6am'').
            \item Lights can be toggled based on day of week (e.g. ``turn on
                during Wednesdays'').
            \item Lights can be toggled based on day of month (e.g. ``turn on
                every 1st of the month'').
            \item Lights can be toggled based on day of year (e.g. ``turn on
                every January 1st'').
            \item Lights can be toggled based on specific dates (e.g. ``turn on
                from Dec. 24th at 2am to Dec 27th at 8pm'').
            \item Lights can be toggled based on sunrise/sunset times (using
                sunset/sunrise times either preloaded onto the Pi or retrieved
                from an Internet service, and using a geograpic location either
                entered by the user or determined through an Internet service)
            \item Multiple rules can be applied to a single light/group using
                an AND/OR system to combine the rules (e.g. ``turn on between
                8pm and 6am AND turn on on Wednesday'' will turn the lights on
                between 8pm and 6am on Wednesdays, and will be off the rest of
                the time, whereas using an OR instead of an AND would cause the
                lights to be on every day from 8pm to 6am and also on all day
                during Wednesday).
            \item Lights can be toggled based on the toggle state of its parent
                group (e.g. by default, a light will turn on if its parent
                group turns on, but it can also be set to turn off if the
                parent group turns on).
            \item Rules can be set to toggle early/late by a constant or random
                period of time.  For example, lights can be configured to turn
                on exactly 30 minutes after sunset, or perhaps randomly between
                6pm and 6:30pm.
            \item Lights within a group can be set to individually toggle
                early/late by a random period of time so that they toggle in a
                staggered fashion.  For example, if there are 5 lights in a
                group and are set to toggle off at 8pm, you can have the first
                light randomly toggle a few seconds before 8pm, then the next a
                few seconds later, etc. in a random order and with randomized
                times.
            \item Lights/groups can be set to gradually dim/brighten over a set
                period of time.  For example, lights can turn on and gradually
                brighten from sunset to 30 minutes after sunset, after which
                they are at full brightness.
        \end{enumerate}
\end{enumerate}

\subsection{Stretch Goals}

The following requirements are optional; they are not critical to the basic
functionality of the system and may be implemented only if time permits.

\begin{enumerate}[resume]
    \item Device control
        \begin{enumerate}
            \item System also controls devices other than lights
                \begin{enumerate}
                    \item Garage door openers (e.g. users can program their
                        garage door to automatically close at night).
                    \item Music players (e.g. users can set music to
                        automatically play in the evenings from an attached
                        music player or from the Internet).
                    \item Holiday decorations (e.g. users can set snowglobes or
                        animatronic deer to automatically turn on at night)
                    \item Sprinkler systems (e.g. users can set their
                        sprinklers to turn on from 4am to 5am).
                \end{enumerate}
        \end{enumerate}
    \item User interface
        \begin{enumerate}
            \item The web interface is accessible over the Internet (i.e. the
                user can control the system from anywhere in the world with an
                Internet connection, not just at their home)
                \begin{enumerate}
                    \item The web interface is accessible only to the homeowner
                        or other authorized individuals (to prevent the general
                        public from being able to view/control the user's
                        lights).  This could be implemented by asking the user
                        to provide a password when logging into the system for
                        the first time from a particular device.  The device
                        could then stay logged in via a cookie until the user
                        logs out.
                    \item The web interface is hosted on a remote server for
                        high availablity and the lack of need for the user to
                        perform port forwarding on their router.  If the web
                        server is running on the user's existing home network,
                        the site won't be available unless the user manually
                        opens their router's interface and creates a port
                        forwarding rule, and also won't be available in the
                        case that the user's router/modem loses power or
                        connection to the Internet.  If the site was hosted
                        externally, the user would still be able to view how
                        the system is currently set up and make changes that
                        will apply once the user's power and/or Internet
                        connection are restored.
                \end{enumerate}
        \end{enumerate}
    \item Rules
        \begin{enumerate}
            \item Lights can be toggled based on weather conditions (e.g.
                lights can turn on automatically if it starts raining).  This
                would require an Internet connection or light/moisture sensors.
            \item Lights can be toggled based on a schedule in the user's
                calendar (e.g. a user could create a lighting schedule on their
                Google calendar and it would ``sync'' with the lighting
                system).  This would require Internet connection.
            \item Lights can be toggled based on moon position or other
                celestial data (e.g. lights could turn off during a full moon
                or solar eclipse for better viewing).  This would likely
                require either an Internet connection or a cache of preloaded
                data and the user's location.
            \item Lights can be toggled based on input from attached sensors
                \begin{enumerate}
                    \item Motion sensors (e.g. lights turn on when someone
                        walks up to the front door)
                    \item Light sensors (e.g. lights turn off at a certain
                        brightness level, regardless of time of day, weather,
                        moon phase, etc.  This would also account for an area's
                        brightness changing with the seasons and tree cover).
                    \item Moisture sensors (e.g. lights turn on when it's wet
                        outside, negating the need for an Internet connection
                        to determine if it's raining andpossibly providing more
                        accurate results.  Could also potentially be used in
                        combination with sprinkler integration to stop watering
                        once a certain moisture level has been reached).
                    \item Sound sensors (e.g. lights turn on after a loud
                        noise, or can strobe according to the beat of music
                        that is playing).
                \end{enumerate}
        \end{enumerate}
\end{enumerate}

\section{Preliminary Timetable}

By the end of fall term, we plan to have a working proof-of-concept, using the
Raspberry Pi and a basic control program to toggle lights on and off.  By the
end of winter term, we plan to have all required features at least partially
implemented, at which our project enters the beta phase.  As we finish up by
ironing out bugs and perhaps completing stretch goals, we will release version
1.0 prior to the expo.

\subsection{Gantt Chart}
Gantt chart available at our project SharePoint website:
\url{https://oregonstateuniversity-my.sharepoint.com/personal/rettigs_oregonstate_edu/capstone22_project/_layouts/15/start.aspx#/Lists/Tasks/gantt.aspx}
\begin{sidewaysfigure}
\includegraphics[width=1.0\textwidth]{ganttchart.png}
\end{sidewaysfigure}

\pagebreak

\section{Signatures}

\begin{tabular}{l l l l} Malcolm Diller & \underline{\hspace{6cm}} & Date
\underline{\hspace{2cm}}\\ Sean Rettig & \underline{\hspace{6cm}} & Date
\underline{\hspace{2cm}}\\ Evan Steele & \underline{\hspace{6cm}} & Date
\underline{\hspace{2cm}}\\ Victor Hsu & \underline{\hspace{6cm}} & Date
\underline{\hspace{2cm}} \end{tabular}
    
\end{document}
