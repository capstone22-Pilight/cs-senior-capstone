\documentclass{article}
\usepackage[utf8]{inputenc}
\usepackage[margin=1.5in]{geometry}

\title{Not exactly the Internet of Things for Outdoor Lighting: Problem Statement}
\author{Oregon State University CS Senior Capstone Group 129\\Malcolm Diller, Sean Rettig, Evan Steele}
\date{\today}

\begin{document}

\maketitle

\section{Project}

Outdoor lighting seems like a simple problem to solve, but the solutions on the
market today are less than ideal.  The standard transformer/timer combos
available in the big box stores are rudimentary and clunky at best (constantly
needing to account for the sunset and sunrise times), but are reasonably
priced. The new-generation smart apps for home automation are flexible and
fancier, but are quite spendy and lock you into a specific protocol.  So why
not use an open platform running on commodity hardware?  Easy to use,
reasonably priced, and highly customizable--that is our goal.

Our system will consist of a wireless network of tiny “client” computers that
each control up to 4 lights and are controlled by a central "server" computer,
which will automatically send out commands to the clients when it's time to
turn on or off.  The central node will run a control program that can be easily
accessed via a touch screen, a web browser, or a mobile device, where the user
can locally or remotely control each light individually.  Want your lights to
turn on at sunset and then dim gradually as the sun rises?  Simple.  Want your
lights to flash when you're throwing a party?  Just press a button.  The
control interface will allow users to easily set "rules" for what their lights
do and when, depending on the time of day, the sun/moon position, and
potentially even triggers such as weather conditions or calendar dates.  This
system will be easily extensible to potentially control other devices as well,
such as garage doors, sound systems, and more.

To start off, this project will focus on a basic program to run on the server
computer that can successfully communicate on/off instructions to the client
computers, which then actually turn the lights on/off.  Next, we will need to
create an interface for the server node that can be accessed either through the
touchscreen or through a web browser/mobile device to actually configure the
system.  The final main component will be the implementation of the rule
system, which will allow lights to be toggled based on external factors such as
sun position.

What will we be showing at expo?  Simple: the whole thing!  All we need is the
system itself and a power source.  Visitors will be able to use the interface
to control the lights at the booth!

\end{document}
